\chapter{Modellierung der Test-Daten}
\label{chap:modellierung}



\section{Anforderungen an die DSL}

Eine der wichtigsten Anforderungen an die DSL ist, dass sie sich in die bestehende Werkzeugkette der Firma Seitenbau integrieren lassen muss.
Daraus folgt die Anforderung, dass sie sich in Java nutzen lassen soll. �hnlich wie bei \textit{SB Testing DB} sollen Datasets auch 
nachtr�glich ver�nderbar sein. 

Die Sprache soll auf syntaktischen Ballast verzichten und einen �bersichtlichen Code zur Modellierung der Daten erm�glichen. Meta-Informationen sollten ausschlie�lich in Form von Sprachelementen auftauchen.

Beziehungen sollen sich einfach und typsicher modellieren lassen. Es soll nicht mehr notwendig sein, symbolische Java-Konstanten z.B. f�r die Definition von ID-Nummern zu verwenden. 



\todo{Anforderungen}
\begin{itemize}
	\item Zugriff auf Daten aus Java heraus
	\item "`Dekorieren"' von Werten (before(date))
	\item Typ-Pr�fungen zu Compilierzeit
	\item Namensr�ume/Scopes
\end{itemize}

\section{Zielgruppe}

- Techniker 
- Im Umgang mit Datenbank versierte Anwender

\section{DSL-Entw�rfe}

	\subsection{Entwurf 1}
	
	Eine DSL, die sich stark an \textit{SB Testing DB} orientiert, k�nnte wie folgt aussehen:
	
	\begin{lstlisting}[caption=M�gliche DSL (1), label=listing:dslentwurf1]
HAASE = professor {
	name			"Haase"
	vorname   "Oliver"
	titel     "Prof. Dr."
  fakultaet "Informatik"
}

WAESCH = professor {
	name			"W�sch"
	vorname   "J�rgen"
	titel     "Prof. Dr.-Ing."
  fakultaet "Informatik"
}
	
VSYS = lehrveranstaltung {
	name			"Verteilte Systeme"
  sws       4
	ects      5
}
	
DPATTERNS = lehrveranstaltung {
	name 			"Design Patterns"
	sws       4
	ects      3
}

...

HAASE leitet VSYS
HAASE leitet DPATTERNS
HAASE beaufsichtigt	P_DPATTERNS
WAESCH beaufsichtigt P_VSYS
...

	\end{lstlisting}
	
	Diese DSL kommt ohne manuell vergebene ID-Nummern aus und verwendet Variablennamen f�r die Modellierung von Beziehungen. Da f�r jeden
	Wert eine eigene Zeile verwendet wird, werden umfangreiche Daten schnell un�bersichtlich. Die Beschreibung der Beziehungen abseits der
	Definition der Daten erschwert den Umgang mit den Daten und die �bersicht ebenfalls.


	\subsection{Entwurf 2}
	
	Ein leicht abgewandelter Entwurf zeigt, wie sich die Beziehungen n�her an den eigentlichen Daten beschreiben lassen k�nnten.
	An dem Problem, dass die Daten relativ schnell in vertikaler Richtung wachsen, �ndert das jedoch nichts.
	

	\begin{lstlisting}[caption=M�gliche DSL (2), label=listing:dslentwurf2]
HAASE = professor {
	name      "Haase"
	vorname   "Oliver"
	titel     "Prof. Dr."
  fakultaet "Informatik"
	leitet    VSYS, DPATTERNS
	beaufsichtigt	P_DPATTERNS
}

WAESCH = professor {
	name      "W�sch"
	vorname   "J�rgen"
	titel     "Prof. Dr.-Ing."
  fakultaet "Informatik"
	beaufsichtigt	P_VSYS
}
	
VSYS = lehrveranstaltung {
	name			"Verteilte Systeme"
  sws       4
	ects      5
}
	
DPATTERNS = lehrveranstaltung {
	name 			"Design Patterns"
  sws       4
	ects      3
}

...
	\end{lstlisting}
	

	\subsection{Entwurf 3}
	
	Der dritte Entwurf versucht die Daten durch eine tabellarische Struktur �bersichtlich zu gestalten. Sie kommt mit
	wenig syntaktischem Ballast aus. 
	
	

	\lstSetTiny
	\begin{lstlisting}[caption=M�gliche DSL (3), label=listing:dslentwurf3]
professor:
REF    || name    | vorname  | titel            | fakultaet    | leitet          | beaufsichtigt
HAASE  || "Haase" | "Oliver" | "Prof. Dr."      | "Informatik" | VSYS, DPATTERNS | P_DPATTERNS   
WAESCH || "W�sch" | "J�rgen" | "Prof. Dr.-Ing." | "Informatik" |                 | P_VSYS
	
lehrveranstaltung:
REF       || name                | sws | ects
VSYS      || "Verteilte Systeme" | 4   | 5
DPATTERNS || "Design Patterns"   | 4   | 3

...
	\end{lstlisting}
	\lstSetNotmal
	
	Probleme bzw. Nachteile in der Darstellung k�nnen auftreten, wenn die L�nge der Werte in einer Spalte stark variiert.
	Der Entwickler ist selbst daf�r verantwortlich, die �bersichtliche Darstellung einzuhalten. Auf Tabulatoren sollte unter 
	Umst�nden verzichtet werden, da sie von verschiedenen Editoren unterschiedlich dargestellt werden k�nnen. Bei vielen Spalten
	w�chst diese Darstellung horizontal. Bei optionalen Spalten bzw. kaum genutzte Spalten kann die tabellarische Darstellung
	un�bersichtlich werden.
	
	
	Besser:
	- Spalten-Editierung mit Block-Bearbeitungsmodus m�glich
	
	Schlecht:
	- Tabellenkopf muss u.U. wiederholt werden, um �bersicht zu erhalten
	- Spaltenbreite nur von l�ngstem Wert abh�ngig (-> Konstanten?)



\section{Implementierung}
\label{sec:modellierung:implementierung}

Da sich die DSL in die bisherige Werkzeug-Kette von Seitenbau integrieren lassen soll
(\refsec{sec:anforderungen:allgemeineanforderungen}), sollte die DSL in Java nutzbar sein.
Grunds�tzlich kann eine DSL auch in Java direkt realisiert werden. Allerdings bietet sich vor allem im
Java-Umfeld die Sprache Groovy an. 

- Groovy verwendet Java-Objekt-Modell
- Meta-Programmiernug
- Closures
- Operator �berladen

Deshalb empfiehlt Ghosh in \cite[148]{DSLS_IN_ACTION} Groovy als Host f�r DSLs in Verbindung
mit Java-Anwendungen.


	\subsection{Implementierungsvarianten}

		\subsubsection{AST-Transformation}

		\subsubsection{Operator-�berladen}
		
