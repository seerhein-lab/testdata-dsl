\chapter{Einleitung}

Softwaretests sind ein wichtiger Baustein f�r die Qualit�tssicherung von modernen Softwareprojekten.
F�r Tests werden Testdaten spezifiziert. Auf deren Basis wird das Verhalten der Software gepr�ft.
Bei Datenbank-basierten Anwendungen werden Testdaten in der Regel sehr umfangreich und komplex. 
In diesem Kontext spricht man auch von der Modellierung von Testdaten.

Die Komplexit�t ergibt sich aus der Beschreibung von Beziehungen zwischen den einzelnen
Datens�tzen. Besonders bei Systemen mit komplexen Datenbank-Schemata kann ein
Testdaten-Modell schnell un�bersichtlich werden.

F�r den Tester sind un�bersichtliche Testdaten aus verschiedenen Gr�nden ein Problem.
Einerseits machen sie die Pflege fehleranf�llig. Andererseits ist es schwer,
die modellierten Daten zu erfassen und zu verstehen. Tester w�nschen deshalb h�ufig Daten,
die f�r mehrere Tests nutzbar sind. Dadurch reicht es aus, nur ein oder zumindest
wenige Daten-Modelle zu verstehen und zu pflegen.


\section{Zielsetzung der Arbeit}

Die Beschreibung von Testdaten soll vereinfacht werden. Dazu soll eine Modellierungssprache
entwickelt werden, die leicht zu verstehen und zu erlernen ist. Besonderer
Fokus liegt auf der Modellierung von Beziehungen. Die L�sung soll einfach zu benutzen sein
und sich in Entwicklungsumgebungen (wie Eclipse) integrieren lassen.

Dar�ber hinaus sollen Testdaten automatisch generiert werden k�nnen. Hier wird besondere
Aufmerksamkeit auf die Generierung von Daten f�r Beziehungen zwischen Datens�tzen gelegt.
Wichtig ist vor allem, dass die erzeugten Daten zum Datenbank-Schema passen, d.h. dass
g�ltige Beziehungen erzeugt werden.


\section{Aufbau der Arbeit}

Zun�chst werden in Kapitel~\ref{chap:grundlagen} einige grundlegende Konzepte bzw. Technologien und
in der Arbeit verwendete Konventionen beschrieben, die f�r das weitere Verst�ndnis notwendig
sind. Danach werden in Kapitel~\ref{chap:anforderungen} die Anforderungen an die
Modellierungssprache erarbeitet und die Aufgabenstellung konkretisiert. Die Entwicklung und die
Definition der Sprache wird in Kapitel~\ref{chap:modellierung} beschrieben. 
Kapitel~\ref{chap:realiserungdsl} beschreibt die Implementierung und schlie�t damit das Thema
Modellierungssprache ab.
Mit der Generierung von Testdaten befasst sich Kapitel~\ref{chap:generieren}.
Abschlie�end wird in Kapitel~\ref{chap:zusammenfassung} eine Zusammenfassung und ein
Ausblick auf m�gliche Folgearbeiten gegeben.


