\chapter{Generieren von Testdaten}
\label{chap:generieren}

Fragen: 
- Wie muss dass Modell "`angereichert"' werden? 
- Wie k�nnen Daten sinnvoll generiert werden?

- Viele Generatoren f�r reine Daten, mit Beziehungen schon weniger


Red Gate SQL Data Generator:
- Fragt Datenbank-Schema aus SQL-Server ab
- Modelliert Beziehungen(!), allerdings zuf�llig und bedeutungslos
- Anzahl generierter Zeilen pro Tabelle steuerbar
- Zufall pro Spalte �ber Seed steuerbar
- Eher f�r gro�e Datenmengen gedacht


DTM Data Generator
- Weniger bequem als Red Gate, z.B. keine Erkennung der Spaltentypen
- 



- Beziehungstypen, wie werden sie generiert
- Assoziative Tabellen


\section{Warum Generierung}
M�glichkeiten:
- Generierung
- Real-Daten, Extraktion, Anonymisierung (bei Unit-Tests wenig praktikabel, da Tests vor Inbetriebnahme laufen sollen)

Je nach Anwendungsfall unterschiedliche Data-Sets sinnvoll
- Unit-Tests eher klein
- Performance-Tests, Regressionstests eher gro�

\section{Generieren von Foreign-Key-Beziehungen}

\subsection{1:1}

\subsection{1:n}

\subsection{n:m}

\section{Erweiterungen am Modell}

\section{Der Algorithmus}
1. Start-Tabelle festlegen
2. wenn Tabelle noch nicht behandelt, dann
3. Kanten bestimmen und behandeln (ausgehende bevorzugt)
4. Schritte 2-4 rekursiv f�r die verbundene Tabelle wiederholen
5. Pr�fen ob alle Entit�ten g�ltige Beziehungen haben, ggf. Entit�ten erstellen

Assoziative Tabellen werden separat behandelt
Jede Kante und jede Tabelle wird nur ein mal in den Schritten 2-4 besucht


\section{Probleme}