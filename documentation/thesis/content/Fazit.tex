\chapter{Zusammenfassung und Ausblick}
\label{chap:zusammenfassung}

Das wesentliche Ziel dieser Arbeit war es, eine einfachere Sprache zur Modellierung
von Testdaten f�r Datenbank-basierte Anwendungen zu entwickeln. Neben einer leichten
Erlernbarkeit sollte sie sich durch eine einfache und �bersichtliche Modellierung
von Beziehungen in Datens�tzen auszeichnen. Dies wurde mit der entwickelte
Modellierungssprache mehr als erreicht.

Die Benutzbarkeit der Sprache und der Implementierung war ein Kernmerkmal, das nie
au�er Acht gelassen wurde. Die Sprache wurde als DSL in Groovy realisiert und l�sst
sich somit im Java-Umfeld nutzen. Vor allem kann sie auch in Unit-Tests verwendet
werden. Dadurch sind die Tests und die Testdaten nicht voneinander getrennt, wie es
bei extern definierten Testdaten der Fall w�re. 

Die Sprache wird passend f�r ein Datenbank-Modell generiert. Das Datenbank-Modell
beschreibt die Tabellen, deren Spalten und die Beziehungen. Anhand dieses Modells
erzeugt ein Generator die DSL.

Um den Testern die Arbeit mit der DSL zu erleichtern, wurden einige Features
implementiert. So k�nnen unterschiedliche DataSets definiert werden, die die selben
Bezeichner f�r Entit�ten verwenden. Diese Bezeichner repr�sentieren je nach
aktivem DataSet eine andere Entit�t und damit m�glicherweise auch unterschiedliche
Daten. Es ist m�glich, DataSets aus anderen DataSets zusammenzusetzen bzw. bestehnde
zu erweitern. Bei Fehlern in DataSet-Definitionen erh�lt der Tester sinnvolle 
Exceptions und kann so schnell die Ursache im Quellcode finden.
Die generierte JavaDoc zu der DSL rundet die Features ab. Sie erleichtert
den Umgang �ber Code-Beispiele und Informationen zum Datenbank-Modell.

Ein weiteres wichtige Ziel der Arbeit stellt die Generierung von Testdaten dar. Es wurde
gezeigt, dass die Generierung zwar kein neues Thema ist, die meisten Werkzeuge aber
auf die Erzeugung von Massen-Daten ausgelegt sind. Es gibt Ans�tze, die sich mit der
Generierung �berschaubarer Daten befassen, doch diese sind meistens Abfrage-orientiert.
Es wurde ein Algorithmus entwickelt, der die Testdaten anhand der definierten
Beziehungstypen im Datenbank-Modell generiert. Die Anzahl der zu generierenden
Entit�ten leitet sich aus den Beziehungen ab, an denen der Entit�tstyp beteiligt ist.

Der Algorithmus wurde implementiert und die erzeugten Daten in praktischen Anwendungen
getestet. Die Daten sind fachlich korrekt, auch wenn auf einen Beweis daf�r verzichtet
wurde.

% Ausblick
Die L�sung unterst�tzt einige besondere Beziehungstypen nicht direkt. So fehlt der
Tool-Support bei reflexiven Beziehungen und teilweise auch bei zirkul�re Beziehungen.
Die Ursache daf�r liegt in den Klassen zum Beschreiben des Datenbank-Modells.
Um die neue Modellierungssprache zu nutzen, wird momentan noch der Groovy-Compiler
ben�tigt. Um diese Abh�ngigkeit f�r die Tests zu vermeiden, k�nnte dieser der
Implementierung hinzugef�gt werden.

Vor allem das Thema Generierung bietet noch Potential f�r weiterf�hrende Arbeiten. So
betrachtet der Algorithmus keine Abh�ngigkeiten zwischen unterschiedlichen Beziehungen,
mit Ausnahme von assoziativen Tabellen. Au�erdem arbeiten die Datengeneratoren f�r
die Spaltenwerte noch komplett unabh�ngig voneinander bzw. von der Entit�t. Es w�re
sinnvoll, den Generator dahingehen zu erweitern, dass er Abh�ngigkeiten von Beziehungen
und auch von Daten ber�cksichtigt.
