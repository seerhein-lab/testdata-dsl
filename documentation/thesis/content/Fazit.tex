\chapter{Zusammenfassung und Ausblick}
\label{chap:zusammenfassung}

Das wesentliche Ziel dieser Arbeit war es, eine einfache Sprache zur Modellierung
von Testdaten f�r Datenbank-basierte Anwendungen zu entwickeln. Die entwickelte
Modellierungssprache hat dieses Ziel mehr als erreicht. Die Sprache zeichnet sich
durch leichte Erlernbarkeit und eine �bersichtliche Beschreibung von Beziehungen
von Datens�tzen aus.

Die Benutzbarkeit der Sprache und der Implementierung war ein Kernmerkmal, das 
durchgehend betrachtet wurde. Die Sprache wurde als DSL in Groovy realisiert und l�sst
sich somit im Java-Umfeld nutzen. Vor allem kann sie auch in Unit-Tests verwendet
werden. Dadurch sind die Tests und die Testdaten nicht voneinander getrennt, wie es
bei Testdaten der Fall w�re, die extern definiert wurden -- also in XML-, CSV- oder
anderen Dateien. 

Die Modellierungssprache wird passend f�r ein Modell der Datenbank generiert. Dieses
Modell beschreibt die Tabellen, deren Spalten und die Beziehungen. Mit diesen
Informationen erzeugt ein Generator ein API als Grundger�st f�r die Sprache.

Um den Testern die Arbeit mit der DSL zu erleichtern, wurden folgende Features
implementiert. So k�nnen unterschiedliche DataSets definiert werden, die dieselben
Bezeichner f�r Entit�ten verwenden. Diese Bezeichner repr�sentieren je nach
aktivem DataSet eine andere Entit�t und damit m�glicherweise auch unterschiedliche
Daten. Die L�sung unterst�tzt die Komposition und das Erweitern von DataSets.
Bei Fehlern in DataSet-Definitionen erh�lt der Tester sinnvolle Fehlermeldungen
und kann schnell die Ursache finden. Die generierte JavaDoc zu der DSL erleichtert
den Umgang mit Hilfe von Code-Beispiele und Informationen zum Datenbank-Modell.

Ein wichtiges Ziel der Arbeit stellte die Generierung von Testdaten dar. Es wurde
gezeigt, dass die Generierung zwar kein neues Thema ist, die meisten Werkzeuge aber
auf die Erzeugung von Massen-Daten ausgelegt sind. Es gibt Ans�tze, die sich mit der
Generierung �berschaubarer Daten befassen. Diese arbeiten meistens Abfrage-orientiert,
d.h. sie erzeugen die Daten auf Basis einer konkreten Datenbankabfrage. 
Es wurde ein Algorithmus entwickelt, der die Testdaten anhand der definierten
Beziehungstypen im Datenbank-Modell generiert. Die Anzahl der zu generierenden
Entit�ten leitet sich aus den Beziehungen ab, an denen der Entit�tstyp beteiligt ist.

Der Algorithmus wurde implementiert und die Nutzbarkeit anhand praktischer Anwendungen
�berpr�ft. Die erzeugten Daten entsprachen den Erwartungen und passten zum
Datenbank-Schema.

% Ausblick
Die L�sung unterst�tzt einige besondere Beziehungstypen nicht direkt. So fehlt die
Unterst�tzung f�r reflexive Beziehungen und teilweise auch f�r zirkul�re Beziehungen.
Die Ursache daf�r liegt in den Klassen zum Beschreiben des Datenbank-Modells.

Das Thema Generierung bietet Potential f�r weiterf�hrende Arbeiten. So
betrachtet der Algorithmus keine Abh�ngigkeiten zwischen unterschiedlichen Beziehungen,
mit Ausnahme von assoziativen Tabellen. Au�erdem arbeiten die Datengeneratoren f�r
die Spaltenwerte noch komplett unabh�ngig voneinander bzw. von der Entit�t. Es w�re
sinnvoll, den Generator dahingehen zu erweitern, dass er Abh�ngigkeiten von Beziehungen
und auch von Daten ber�cksichtigt.

% Nutzen f�r Seitenbau
\todo{Praktischer Nutzen f�r Seitenbau}