\documentclass[
  a4paper,
  twoside,
% oneside,
% openany,
  notitlepage,      %% Separate Titelseite (alternativ: titlepage)
  ngerman,          %% Das Dokument ist in neuen deutschen Rechtschreibung gehalten
  11pt,
% draft,            %% Finden von �berlangen Zeilen (dargestellt durch schwarze Bl�cke)
]{book}             %% Dokumentklasse (alternativ: article, report, etc.)
       
%% \linebreak f�r Zeilenumbruch
\usepackage[latin9]{inputenc}               %% Eingabezeichensatz zur Darstellung von Umlauten (latin1) erweitert um Sonderzeichen wie EURO (latin9)
\usepackage{csquotes}                       %%
\usepackage{times}                          %% Schrift Times
%\usepackage{mathpazo}                      %% Schrift Palatino
%\usepackage{luximono}                      %% Fette Schreibmaschinenschrift
\usepackage[a4paper]{geometry}              %%
\usepackage[hyphens]{url}                   %%
\usepackage{caption}
\usepackage{subcaption}                     %% Besser als subfigure und subfig
\usepackage{picins}                         %%
\usepackage{longtable}                      %%
\usepackage{shadethm}                       %%
\usepackage{amsmath}                        %%
\usepackage{framed, color}                  %%
\usepackage{fancyhdr}                       %%
\usepackage{lscape}                         %% stellenweises Querformat
\usepackage[T1]{fontenc}                    %% Europaeische Schriftcodierung (T1) und 
\usepackage{here}                           %% Grafik dort positionieren wo man es will ([H])Kopfzeile und Seitenzahlen
\usepackage[ngerman]{babel}                 %% => ngerman = neue deutsche Rechtschreibung
\usepackage{graphicx}                       %% Zur Einbindung von Grafiken. Z.B. ueber \includegraphics
\usepackage{listings}                       %% Codelistungs formatieren
\usepackage{chngpage}                       %% Seiteneinstellungen anpassen (adjustwidth)
\usepackage{paralist}                       %% flexible Listen
\usepackage{url}                            %% Darstellung von Internetadressen und Pfaden
\usepackage{setspace}                       %% zum Erzeugen von groesseren Zeilenabstaenden
\usepackage{nomencl}                        %% Abkuerzungsverzeichnis erstellen
\usepackage{tabularx}                       %% Tabellen mit fester breite nutzen
\usepackage{multirow}                       %% Zellen in Tabellen zusammenfassen
\usepackage[clearempty]{titlesec}           %% Keine Seitenzahlen, Kopf- und Fusszeilen auf leeren Seiten
\usepackage{savefnmark}                     %% Fussnoten mehrfach verwenden
\usepackage{colortbl}                       %% Farben in Tabellen
\usepackage{booktabs}                       %%
%\usepackage{hyperref}                       %% Phantomsection
\usepackage[backend=bibtex8]{biblatex}      %% "`Abgerufen am"' bei online quellen
\usepackage{eurosym}
\usepackage{placeins}                       %% floatbarrier
\usepackage{blindtext}                      %% lorem ipsumtext
\usepackage{nomencl}                        %% Abk�rzungsverzeichnis
%\usepackage[intoc]{nomencl}                 %% Abk�rzungsverzeichnis
\usepackage{xspace}                         %% nicht erzwungene Leerzeichen (f�r \newcommand n�tzlich}
\usepackage{todo}

%\onehalfspacing                             %% 1.5 Zeilenabstand
%\doublespacing                              %% Doppelter Zeilenabstand
%%%%%%%%%%%%%%%%% links %%%%%%%%%%%%%%%%%%%%%%%%%%%%%%%%%%%%%%%%%%
\usepackage[pdftex, pdfborder={0 0 0},colorlinks, linkcolor=black, citecolor=black, filecolor=black, urlcolor=blue, bookmarks=true, bookmarksnumbered=true]{hyperref}
%%%%%%%%%%%%%%%%%%%%%%%%%%%%%%%%%%%%%%%%%%%%%%%%%%%%%%%%%%%%%%%%%%

%-------------------------------------------------------------------------
%���Listings
%-------------------------------------------------------------------------

% Einstellungen f�r Listings mit Eclipse Code Style.
\usepackage{color}
\usepackage{courier}
\lstset{language=java}
\definecolor{sh_comment}{rgb}{0.12, 0.38, 0.18 }     % adjusted, in Eclipse: {0.25, 0.42, 0.30 } = #3F6A4D
\definecolor{sh_keyword}{rgb}{0.37, 0.08, 0.25}      % #5F1441
\definecolor{sh_string}{rgb}{0.06, 0.10, 0.98}       % #101AF9% Farbe f�r Kommentare
\def\lstsmallmath{\leavevmode\ifmmode \scriptstyle \else� \fi}
\def\lstsmallmathend{\leavevmode\ifmmode� \else� \fi}
\lstset {
  frame=single, %shadowbox,
  rulesepcolor=\color{black},
  showspaces=false,
	showtabs=false,
	tabsize=2,
  numberstyle=\tiny\ttfamily,
	numbers=left,
  basicstyle=\scriptsize\ttfamily,
  stringstyle=\color{sh_string},
  keywordstyle=\color{sh_keyword}\bfseries,
  commentstyle=\color{sh_comment}\itshape,
  captionpos=b,
  xleftmargin=0.7cm,
	xrightmargin=0.5cm,
  lineskip=-0.1em,
	breaklines=true,
  escapebegin={\lstsmallmath},
	escapeend={\lstsmallmathend}
}
\newcommand{\lstJava}[1]{\lstinline[language=Java,breaklines=true,basicstyle= \listingsfontinline]$#1$}
\newcommand{\lstXML}[1]{\lstinline[language=XML,breaklines=true,basicstyle= \listingsfontinline]$#1$}

\newcommand{\lstSetTiny}{\lstset{basicstyle= \tiny\ttfamily}}
\newcommand{\lstSetNotmal}{\lstset{basicstyle= \scriptsize\ttfamily}}
\newcommand{\lstSetBig}{\lstset{basicstyle= \ttfamily\mdseries}}

\newcommand{\lstSetJava}{\lstset{language=java}}

\newcommand{\lstSetXML}{\lstset{
  language=XML,
  morekeywords={encoding,
    xs:schema,xs:element,xs:complexType,xs:sequence,xs:attribute}
}}

%\usepackage{fontspec}                        %% Codelistungs formatieren
%\newfontfamily\listingsfontinline[Scale=0.8]{Courier New} 
%\newfontfamily\listingsfont[Scale=0.7]{Courier}


\bibliography{bibliography/bibliography}
\definecolor{shadecolor}{gray}{.65}
\definecolor{lightgrey}{gray}{.90}
\geometry{top=3cm, bottom=3cm, inner=4cm, outer=3cm}  %% Seitengroesse festlegen 

%%Einr�ckung bei neuem Absatz entfernen
\setlength{\parindent}{0pt}
\setlength{\parskip}\medskipamount % besser als explizite Angabe in pt

%%%%%%%%%%%%%%%%  Kopfzeile und Seitenzahlen %%%%%%%%%%%%%%%%%%%%%
\setlength{\headheight}{27pt}    %\setlength{\headheight}{15pt}
\pagestyle{fancy}
\fancypagestyle{plain}{}
\lhead{} 
\fancyfoot{}                      % Fu�zeile l�schen?
\fancyfoot[EL]{\thepage}          % Seitenzahlen rechts anzeigen auf geraden seiten
\fancyfoot[OR]{\thepage}
\renewcommand{\headrulewidth}{0pt}
%%%%%%%%%%%%%%%%%%%%%%%%%%%%%%%%%%%%%%%%%%%%%%%%%%%%%%%%%%%%%%%%%%

%%%%%%%%%%%%%%%%%%% Abk�rzungsverzeichnis %%%%%%%%%%%%%%%%%%%%%%%%
\let\abbrev\nomenclature
\renewcommand{\nomname}{Abk�rzungsverzeichnis}
\setlength{\nomlabelwidth}{.25\hsize}
\renewcommand{\nomlabel}[1]{#1 \dotfill}
\setlength{\nomitemsep}{-\parsep}
\makeindex
\makenomenclature
%%%%%%%%%%%%%%%%%%%%%%%%%%%%%%%%%%%%%%%%%%%%%%%%%%%%%%%%%%%%%%%%%%


%% Anzahl der Nummerierungsebenen festlegen
%% In der book-Klasse wird nur bis zur 3 Ueberschirftenebene
%% durchnummeriert. M�chte man weitere Ebenen hinzufuegen, so
%% muss dies hier angegeben werden. Z�hlung beginnt bei 0.
\setcounter{tocdepth}{2} % im Inhaltsverzeichnis anzeigen
\setcounter{secnumdepth}{2}

%% Hurenkinder und Schusterjungen vermeiden
\widowpenalty=10000
\clubpenalty=10000

% automatischen Zeilenumbruch verbessern
\setlength{\emergencystretch}{10em}

%farben interessant \lstset{language=Java,captionpos=b,tabsize=3,frame=lines,keywordstyle=\color{blue},commentstyle=\color{darkgreen},stringstyle=\color{red},numbers=left,numberstyle=\tiny,numbersep=5pt,breaklines=true,showstringspaces=false,basicstyle=\footnotesize,emph={label}}

%% 
%% Sollte es vorkommen das Woerter nicht korrekt getrennt werden, bzw. nur an bestimmten Stellen oder nie getrennt werden sollen,
%% so koennen diese, durch Leerzeichen getrennt, mit der richtigen Trennung (durch ein Minus dargestellt) hier angegeben werden.
\include{praeambel/hyphenation}

\newcommand{\akademGradKurz}{Master}
\newcommand{\akademGrad}{\akademGradKurz~of Science}
\newcommand{\ausarbeitung}{\akademGradKurz-Arbeit}
\newcommand{\thema}{Modellierung von Testdaten}
\newcommand{\abgabedatum}{11. Oktober 2013}
\newcommand{\ort}{Konstanz}
\newcommand{\studGang}{Master Informatik}

\newcommand{\geboren}{22.12.1981}
\newcommand{\geborenIn}{TODO}
\newcommand{\autor}{Nikolaus Moll}
\newcommand{\autorMatNr}{287336}
\newcommand{\autorStrasse}{TODO}
\newcommand{\autorPLZ}{TODO}
\newcommand{\autorOrt}{TODO}

\newcommand{\prueferA}{TODO}
\newcommand{\prueferATitle}{TODO}
\newcommand{\prueferAInstanz}{TODO}
\newcommand{\prueferAAdresse}{TODO}
\newcommand{\prueferAPLZ}{TODO}
\newcommand{\prueferAOrt}{TODO}

\newcommand{\prueferB}{PRUEFERB}
\newcommand{\prueferBTitle}{PRUEFERBTITLE}
\newcommand{\prueferBInstanz}{TODO}
\newcommand{\prueferBAdresse}{TODO}
\newcommand{\prueferBPLZ}{TODO}
\newcommand{\prueferBOrt}{TODO}

\newcommand{\footnoteremember}[2]{
  \footnote{#2}
  \newcounter{#1}
  \setcounter{#1}{\value{footnote}}

} \newcommand{\footnoterecall}[1]{

  \footnotemark[\value{#1}]
} 

\newcommand{\refabb}[1]{siehe Abb.~\ref{#1}}
\newcommand{\refsec}[1]{siehe Abschnitt~\ref{#1}}
\newcommand{\refchap}[1]{siehe Kapitel~\ref{#1}}
\newcommand{\reflst}[1]{siehe Listing~\ref{#1}}

\begin{document}

  %% erzeugt zunaechst roemische Seitezahlen
  %\pagenumbering{roman}
  %% bei book-Klassen steht dafuer der Befehl \frontmatter zur Verfuegung
  \frontmatter
  \begin{titlepage}

% Logo einbinden
\includegraphics[width=13cm]{images/cover/htwg_logo.pdf} \\[0.5cm]
\\[3.5cm]
\begin{center}	

	% Schriftgroesse auf maximal setzen
	\Huge{
	
		% Fettschrift mit breitem Verlauf
		\textbf{\thema} \\[7.5cm]
	}
	
%	\centering
%		\includegraphics[width=8cm]{images/cover.png} \\[3.5cm]
	
	\LARGE{
		\textbf{\autor}\\
	}
	\autorMatNr \\
		
	\LARGE{
		\ort, \abgabedatum \\[1.5cm]
	}
	
	\Huge{
		\textbf{\ausarbeitung}
	}

\end{center}

\end{titlepage}
  \newpage
  \thispagestyle{empty}~  

  \thispagestyle{empty}
{
\setlength{\parskip}{0.5cm}
        \begin{center}
        \textbf{\huge \ausarbeitung}

        \textbf{zur Erlangung des akademischen Grades}

        \textbf{\Large \akademGrad}

        \textbf{an der}

        \textbf{Hochschule Konstanz}\\
        Hochschule f�r Technik, Wirtschaft und Gestaltung

        Fakult�t Informatik\\
        Studiengang \studGang
        \end{center}
}
\begin{center}
\begin{verbatim}


\end{verbatim}
\begin{tabular}{p{3cm}p{10cm}}
\textbf{Thema:} & \textbf{\thema} \\
 & \\
 & \\
 & \\
\textbf{Verfasser:} & \autor \\
	& \autorStrasse \\
	& \autorPLZ\ \autorOrt \\
	& \\
\textbf{1. Pr�fer:} & \prueferATitle\ \prueferA \\
	& \prueferAInstanz \\
	& \prueferAAdresse \\
	& \prueferAPLZ\ \prueferAOrt \\
	& \\
\textbf{2. Pr�fer:} & \prueferBTitle\ \prueferB \\
	& \prueferBInstanz \\
	& \prueferBAdresse \\
	& \prueferBPLZ\ \prueferBOrt \\
	& \\
\textbf{Abgabedatum:} & \abgabedatum
\end{tabular}
\end{center}
  \newpage
  \thispagestyle{empty}~
  
  \chapter{Abstract}
%\addcontentsline{toc}{chapter}{Abstract (Zusammenfassung)}

\begin{center}
	\begin{tabular}{p{3cm}p{10cm}}
		\textbf{Thema:} & \thema \\
		 & \\
		\textbf{Verfasser:} & \autor \\
		 & \\
		\textbf{Betreuer:} & \prueferATitle\ \prueferA \\
		                   & \prueferBTitle\ \prueferB \\
		 & \\
		\textbf{Abgabedatum:} & \abgabedatum \\
		 & \\
		\textbf{Schlagworte:} & Modellierung von Testdaten, Beziehungen in Testdaten, Generieren von Testdaten, DSL f�r Testdaten, Datenbanktest \\
		 & \\
	\end{tabular}
\end{center}

Softwaretests haben sich als Teil der Qualit�tssicherung von Softwareprojekten etabliert.
F�r Tester ist die Modellierung von Testdaten f�r Datenbank-basierte Anwendungen allerdings
nicht immer einfach. Die Daten k�nnen aufgrund von Beziehungen von Datens�tzen schnell
un�bersichtlich und komplex werden. Wegen der Komplexit�t versuchen Tester, mehrere Tests
mit denselben Daten durchzuf�hren.

In dieser Masterarbeit werden eine neue Modellierungssprache f�r Testdaten f�r
Datenbank-basierte Anwendungen und ein Algorithmus zur Generierung von Testdaten 
vorgestellt. Die Sprache erlaubt eine �bersichtliche Beschreibung von Daten und
von Beziehungen zwischen Datens�tzen. Der Algorithmus zur Generierung erzeugt
Daten anhand der Beziehungstypen im Datenbank-Modell. Der Algorithmus versucht
viele Grenzf�lle zu erzeugen, so dass die Daten in m�glichst vielen Tests verwendet
werden k�nnen.

Die Sprache und der Algorithmus wurden implementiert und in praktischen Anwendungen
auf ihre Nutzbarkeit �berpr�ft. Die Implementierung steht unter der Open-Source-Lizenz
\textit{Apache License 2.0}.





\todo{Bild Architektur tauschen}

\todo{auf externe Daten eingehen}

  \newpage
  \thispagestyle{empty}~
  
  \chapter*{Ehrenw�rtliche Erkl�rung}
\addcontentsline{toc}{chapter}{Ehrenw�rtliche Erkl�rung}

Hiermit erkl�re ich
\textit{\autor,} geboren am \textit{\geboren} in \textit{\geborenIn,} dass ich\\

\begin{tabular}{lp{12cm}}
(1) & meine \ausarbeitung\ mit dem Titel \\[1em]
& \textbf{\thema} \\[1em]
& selbst�ndig und ohne fremde Hilfe angefertigt und keine anderen als die angef�hrten Hilfen benutzt habe;\\[1em]
(2) & die �bernahme w�rtlicher Zitate, von Tabellen, Zeichnungen, Bildern und
Programmen aus der Literatur oder anderen Quellen (Internet) sowie die Verwendung
der Gedanken anderer Autoren an den entsprechenden Stellen innerhalb der Arbeit
gekennzeichnet habe.\\
\end{tabular}

\vspace*{1cm}

Ich bin mir bewusst, dass eine falsche Erkl�rung rechtliche Folgen haben kann.\\

\vspace*{2cm}

\ort, \abgabedatum

\begin{verbatim}



\end{verbatim}

\begin{tabular}{c}
  \rule{5cm}{1pt}\\
  \autor\\
\end{tabular}
  \newpage
  \thispagestyle{empty}~  
  
  %% Inhaltsverzeichnis erstellen
  \cleardoublepage
  \phantomsection % n�tig damit das Inhaltsverzeichnis in selbigem korrekt verlinkt wird und keine Warnung geworfen wird
  %\addcontentsline{toc}{chapter}{Inhaltsverzeichnis}
  \tableofcontents

  %% Setzt den Seitezaehler zurueck und erzeugt ab hier arabische Seitenzahlen
  %\pagenumbering{arabic}
  %% bei book-Klassen steht dafuer der Befehl \mainmatter zur Verfuegung
  \mainmatter
  
  \chapter{Einleitung}

Softwaretests sind ein Grundpfeiler f�r die Qualit�tssicherung von Anwendungen. 
Um Datenbank-basierte Anwendungen zu testen, m�ssen Test-Daten spezifiziert werden. Man
spricht hier auch von der Modellierung von Test-Daten.

Eine Herausforderung bei der Modellierung stellt nicht nur das Spezifizieren der Daten,
sondern auch das Ausdr�cken von Beziehungen von Datens�tzen dar. Besonders bei Systemen mit 
komplexen Tabellen-Schemata kann die Modellierung un�bersichtlich werden.

F�r den Tester ist eine un�bersichtliche Modellierung aus verschiedenen Gr�nden ein Problem.
Einerseits macht sie die Modellierung fehleranf�lliger. Andererseits ist es schwerer,
die modellierten Daten zu erfassen und zu verstehen. Oftmals werden deshalb f�r verschiedene
Tests dieselben Daten verwendet.

 
- Generierung von Daten mit Fokus auf Beziehungen



\todo{Einleitung schreiben}
- Motivation / allgemeine Problemstellung

- Aufgabenstelung / Zielsetzung


\section{Zielsetzung}

\section{Aufbau der Arbeit}


\todo{Weitere Quellen und Zitate einbauen}

\section*{Ungenutzte Quellen}
\begin{enumerate}
	\item \cite[20ff]{DER_INTEGRATIONSTEST}
	\item \cite{MODELLGETRIEBENE_SOFTWAREENTWICKLUNG}
	\item \cite{DOMAIN_DRIVEN_DESIGN}
\end{enumerate}

  \chapter{Grundlagen}
\label{chap:grundlagen}

\section{Fortlaufendes Beispiel}
\label{sec:grundlagen:beispiel}

% Beispiel einleiten
Eine einheitliche und fortlaufende Problemstellung soll der Arbeit als Grundlage dienen. Die Problemstellung besteht aus einem Modell und einem Satz von Testdaten. Alle im weiteren Verlauf diskutierten Modellierungsvarianten werden  diese Problemstellung umsetzen und die Testdaten modellieren.  

	\subsection{Voraussetzungen}
	\label{sec:grundlagen:beispiel:voaussetzungen}

	Der Schwerpunkt der Modellierung liegt bei der Darstellung von Beziehungen zwischen Entit�ten. Dabei soll die
	Problemstellung einerseits nicht zu komplex sein, damit sie �berschaubar bleibt. Andererseits soll sie komplex genug
	sein, um m�glichst alle Bezeihungensarten zwischen Entit�ten abzudecken.

	Die Testdaten sollten so gew�hlt werden, dass idealerweise f�r alle Tests die selben Daten verwendet werden k�nnen.
	Einheitiche Daten sorgen daf�r, dass sich der Entwickler (verbessern) nicht bei verschiedenen Tests in unterschiedliche
	Testdaten hineinversetzten muss. Nur in Ausnahmef�llen sollten Tests modifizierte oder eigene Testdaten verwenden.
	Um dem Entwickler entgegen zu kommen, sollte der Umfang der Testdaten nicht gr��er sein als erforderlich.

	\subsection{Gew�hlte Problemstellung}
	\label{sec:grundlagen:beispiel:gewaehlte_problemstellung}
	Das gew�hlte Beispiel stellt eine starke Vereinfachung des Pr�fungswesens an Hochschulen dar. Auf eine praxisnahe
	Umsetzung wird zugunsten der Komplexit�t verzichtet. Es beinhaltet die folgenden vier Entit�ten:

	\begin{itemize}
		\item \textbf{Professor}: Ein Professor leitet Lehrveranstaltugnen.
		\item \textbf{Lehrveranstaltung}: Eine Lehrveranstaltng wird von einem Professor geleitet. Es kann zu jeder
			Lehrveranstaltung eine Pr�fung geben.
		\item \textbf{Pr�fung}: Eine Pr�fung ist einer Lehrveranstaltung zugeordnet. Au�erdem hat mindestens ein Pofessor
			Aufsicht.
		\item \textbf{Student}: Studenten k�nnen an Lehrveranstaltungen und an Pr�fungen teilnehmen. Studenten haben au�erdem 
			die M�glichkeit, Tutoren von Lehrveranstaltugen zu sein.
	\end{itemize}

	Die Beziehungen der Entit�ten stellen sich wie folgt dar: Eine Lehrveranstaltung muss von genau einem Professor
	geleitet werden, ein Professor kann beliebig viele (also auch keine) Lehrveranstaltungen leiten. Eine Pr�fung ist genau
	einer Lehrveranstaltung zugeordnet. Eine Lehrveranstaltung kann mehrere Pr�fungen haben (z.B. Nachschreibepr�fung).
	Eine Pr�fung muss mindestens von einem Professor beaufsichtigt werden, ein Professor kann in beliebig vielen Pr�fungen
	Aufsicht haben. Jeder Student kann beliebig vielen Lehrveranstaltungen besuchen und Lehrveranstalungen von beliebig
	vielen Studenten besucht werden. Die gleiche Bezeihung gilt f�r Tutoren: Jeder Student kann bei beliebig vielen
	Lehrveranstaltungen Tutor sein und jede Lehrveranstaltung kann beliebig viele Tutoren haben. Schlie�lich kann jeder
	Student auch an beliebig vielen Pr�fungen teilnehmen und umgekehrt eine Pr�fung von einer beliebigen Anzahl von
	Studenten geschrieben werden.

	Abbildung \ref{img:example_er} stellt die Problemstellung grafisch dar. Die Abbildung zeigt, dass es keine 1:1
	Beziehung gibt. Eine 1:1-Beziehung kann jedoch als Spezialfall einer 1:n-Beziehung angesehen werden.

	\begin{figure}[H]
		\centering
		 \includegraphics[width=0.8\textwidth]{images/grundlagen/example_hochschule_er.pdf}
		\caption{ER-Diagramm des fortlaufenden Beispiels}\label{img:example_er}
	\end{figure}

	Das entsprechende relationale Modell sieht folgenderma�en aus:
	\begin{figure}[H]
		\centering
		 \includegraphics[width=0.95\textwidth]{images/grundlagen/example_hochschule_relational.pdf}
		\caption{Relationales Modell des fortlaufenden Beispiels}\label{img:example_relational}
	\end{figure}

	\subsection{Wahl der Testdaten}
	\label{sec:grundlagen:beispiel:testdaten}

	Um den einen Kompromiss f�r die Komplexit�t der Testdaten zu finden, werden vier Fragestellungen definiert. Diese
	Fragen sollen dabei helfen, den Umfang der Testdaten bestimmen zu k�nnen. Die Fragen stellen sich wie folgt dar:

	\begin{enumerate}
		\item Welcher Professor unterrichtet die meisten Studenten?
		\item Welcher Student nimmt an den meisten Pr�fungen teil?
		\item Welcher Student ist Tutor und nimmt gleichzeitig an der Pr�fung teil?
		\item Welcher Professor macht die wenigste Aufsicht in Fremdveranstaltungen (Lehrveranstaltungen eines anderen
			Professors)?
	\end{enumerate}


\section{Modellierung der Testdaten in DBUnit}

	\subsection{XML Dataset}

	\lstSetXML
	\begin{lstlisting}[caption=XML Dataset, label=listing:xmldataset]
<!DOCTYPE dataset SYSTEM "dataset.dtd">
<dataset>
  <table name="professor">
	  <column>ID</column>
    <column>name</column>
    <row>
      <value>1</value>
      <value>J�rgen W�sch</value>
    </row>
    <row>
      <value>2</value>
      <value>Oliver Haase</value>
    </row>
    </table>
    <table name="lehrveranstaltung">
    <column>ID</column>
    <column>professorID</column>
    <column>name</column>
    <row>
      <value>1</value>
      <value>2</value>
      <value>Verteilte Systeme</value>
    </row>
    <row>
      <value>2</value>
      <value>2</value>
      <value>Concurrency and Design Patterns</value>
    </row>
  </table>
	\end{lstlisting}

	\subsection{Java Dataset}

	\lstSetJava
	\begin{lstlisting}[caption=Java Dataset, label=listing:javadataset]
DefaultTable professor = new DefaultTable(
    "professor",
    new Column[] { 
      new Column("id", DataType.INTEGER),
      new Column("name", DataType.VARCHAR),
		}
  );
professor.addRow(new Object[] { 
    Parameters.Professor.WAESCH_ID,
    "J�rgen W�sch" 
  });
professor.addRow(new Object[] { 
    Parameters.Professor.HAASE_ID,
    "Oliver Haase"
  });
dataSet.addTable(professor);

DefaultTable lehrveranstaltung = new DefaultTable(
    "lehrveranstaltung", 
    new Column[] {
      new Column("id", DataType.INTEGER),
      new Column("professorid", DataType.INTEGER),
      new Column("name", DataType.VARCHAR), 
    }
  );
lehrveranstaltung.addRow(new Object[] {
    Parameters.Lehrveranstaltung.VERTEILTE_SYSTEME_ID,
    Parameters.Professor.HAASE_ID, 
		"Verteilte Systeme" 
  });
lehrveranstaltung.addRow(new Object[] {
    Parameters.Lehrveranstaltung.DESIGN_PATTERNS_ID,
    Parameters.Professor.HAASE_ID,
    "Concurrency and Design Patterns" 
  });
dataSet.addTable(lehrveranstaltung);
	\end{lstlisting}


	\subsection{SB Testing DSL}

	\begin{lstlisting}[caption=SB Testing Dataset (1), label=listing:sbtestingdataset_old]
table_Professor
  .insertRow()
	   .setId(Parameters.Professor.HAASE_ID)
     .setName("Oliver Haase")
	.insertRow()
    .setId(Parameters.Professor.WAESCH_ID)
    .setName("J�rgen W�sch");
	
table_Lehrveranstaltung
  .insertRow()
    .setId(Parameters.Lehrveranstaltung.VERTEILTE_SYSTEME_ID)
    .setProfessorId(Parameters.Professor.HAASE_ID)
    .setName("Verteilte Systeme")
  .insertRow()
    .setId(Parameters.Lehrveranstaltung.DESIGN_PATTERNS_ID)
    .setProfessorId(Parameters.Professor.HAASE_ID)
    .setName("Design Patterns");	
	\end{lstlisting}


\section{Anforderungen an die DSL}

Die Testdaten sollen in einer \textit{Domain Specific Language} (DSL) beschrieben werden. 


\subsection{Zielgruppe}




  \chapter{Anforderungsanalyse / Fragestellung}
\label{chap:anforderungen}

\todo{Anforderungen Einleiten: Zwei Fragestellungen, DSL und Generierung}

\section{Allgemeine Anforderungen}
\label{sec:anforderungen:allgemeineanforderungen}

Die Hauptziele dieser Arbeit stellen sich wie folgt dar:
\begin{enumerate}
	\item Vereinfachen der Modellierung von Beziehungen
	\item Modellierung von Test-Daten �bersichtlicher machen
	\item Automatisches Generieren von Test-Daten
\end{enumerate}


F�r die Modellierung gelten diese allgemeinen Anforderungen:

\begin{itemize}
	\item \textbf{Integration in bestehende Werkzeugkette}: Die L�sung sollte sich nach M�glichkeit in die bestehende
	  Werkzeugkette von SEITENBAU integrieren lassen.  
		
  \item \textbf{IDE-Integration}: Bedienbarkeit f�r den Tester stellt eine der wichtigsten Anforderungen dar. Daten sollen
	  komfortabel modelliert werden k�nnen. Die Integration in Entwicklungsumgebungen wie Eclipse oder IntelliJ IDEA muss
		gegeben sein. 
	
	\item \textbf{Beziehungen}: Beziehungen sollen einfach modellieren werden k�nnen. 

	\item \textbf{G�ltigkeitsbereiche}:
	  \todo{G�ltigkeitsbereiche erkl�ren}

	\item \textbf{Ver�nderbarkeit von DataSets}: DataSets sollen sich bei der Modellierung beliebig ver�ndern lassen.
	
	\item \textbf{Komposition}: DataSets sollen sich aus anderen DataSets zusammensetzen lassen.
	
	
	\item \textbf{Typ-Sicherheit}: Die Beschreibung der Daten sollte typsicher erfolgen. Idealerweise sollten falsche
	  Typen schon w�hrend des Compilierns erkannt werden.
		
	\item \textbf{Funktionen als Werte}: Es soll m�glich sein, Hilfsfunktionen zur Berechnung von Werten zu verwenden,
	  z.B. zum Einlesen von Binary Large Objects (BLOBs) aus Dateien.
		
	\item \textbf{Zielgruppe}: Die Zielgruppe f�r die DSL sind Software-Entwickler und Tester. Der Code zur Modellierung
	  der Daten sollte auch f�r andere Projekt-Mitglieder lesbar und verst�ndlich sein.

	\item \textbf{Ung�ltige Daten}: Es sollen sich auch aus Sicht der Datenbank oder des SUT ung�ltige Daten modellieren lassen.

\end{itemize}


F�r die Generierung der Testdaten lassen sich die Anforderungen folgenderma�en zusammenfassen:
\begin{itemize}
  
	\item \textbf{Fokus auf Beziehungen}: Das Generieren von sinnvollen Beziehungen stellt eines der zentralen Ziele f�r
	  den Daten-Generator dar.
	
	\item \textbf{Datenmenge selbst bestimmen}: Der Generator soll ohne Konfigurationsaufwand eine geeignete Menge an Test-Daten
	  erzeugen.

	\item \textbf{Deterministische Generierung}: Auch wenn die Test-Daten aus Zufallsdaten bestehen, sollen sie deterministisch
	  generiert werden k�nnen. Das hei�t, dass die Generierung des Modells mit den selben Einstellungen auch zum selben Ergebnis
		f�hrt.
		
	\item \textbf{Kompatibilit�t}: Die Generierung der Testdaten soll in unterschiedliche Ausgabe-Formen erfolgen k�nnen,
	  z.B. in einer DSL, in XML oder auch in SQL-Statements.
	
\end{itemize}


\section{Modellierungskonzepte f�r Beziehungen}
\label{sec:fragestellung:modellierungskonzepte}
	
Je nach Beziehungsart gibt es unterschiedliche Ans�tze, wie diese in einem ER-Diagramm umgesetzt werden k�nnen.
Dabei k�nnen neben den Entit�ten auch die Beziehungen selbst Attribute haben.
Die folgenden drei grunds�tzlichen Beziehungsarten werden dabei unterschieden:

	\subsection{1:1-Beziehungen}
	\label{sec:fragestellung:onetoone}
	
	Eine bin�re Beziehung zwischen zwei Entit�tstypen, wobei jede Entit�t innerhalb dieser Beziehung maximal einer
	anderen Entit�t zugeordnet sein kann. Eine solche Beziehung kann realisiert werden, indem eine Tabelle um einen
	Fremdschl�ssel auf die andere erweitert wird. Dabei sollte der Fremdschl�ssel und auch die beziehungsbeschreibenden
	Attribute immer der Tabelle hinzugef�gt werden, deren Entit�ten eine Beziehung voraussetzt.
	
	Wenn viele Beziehungsattribute vorhanden sind oder die Beziehung auf beiden Seiten optional ist,
	kann es auch sinnvoll sein, eine 1:1-Beziehung wie eine n:m-Beziehung zu modellieren.

	\subsection{1:n-Beziehungen}
	\label{sec:fragestellung:onetomany}

	Eine bin�re Beziehung zwischen zwei Entit�tstypen, wobei jede Entit�t des einen Typs in Beziehung mit mehreren
	Entit�ten des anderen Typs stehen kann. Diese Entit�ten k�nnen auch nur mit maximal einer Entit�t in Beziehung
	stehen. Es ist m�glich festzulegen, wie viele Beziehungen eine Entit�t mindestens und h�chstens haben darf.
	
	Die Tabelle der Entit�ten, die maximal einer andere Entit�t zugeordnet sind, wird um einen Fremdschl�ssel
	und um f�r jede Beziehung individueller Attribute erweitert. Die Beziehungsattribute, die f�r alle Beziehungen
	der beteiligten Entit�t gelten, werden ihrer Tabelle hinzugef�gt.
	
	\subsection{n:m-Beziehungen}
	\label{sec:fragestellung:manytomany}
	
	Eine bin�re Beziehung zwischen zwei Entit�tstypen, wobei jede Entit�t des einen Typs mit mehreren Entit�ten
	des anderen Typs in Beziehung stehen kann -- und umgekehrt. Es ist m�glich, untere und obere Grenzwerte f�r
	die Anzahl der Beziehungen auf beiden Seiten festzulegen. Solche als assoziativ bezeichneten Beziehungen
	werden �ber eine Hilfstabelle modelliert, die entsprechend assoziative Tabelle genannt wird. Diese besteht
	aus den beiden Fremdschl�sseln auf die beteiligten Tabellen und den beziehungsbeschreibenden Attributen.
	
	Grunds�tzlich k�nnen assoziative Tabellen f�r alle bin�ren Beziehungen verwendet werden. Vor allem wenn 
	die Beziehung viele Attribute enth�lt, kann eine assoziative Tabelle f�r �bersichtlichere Tabellenstrukturen
	sorgen.  

	\subsection{Andere Beziehungen}
	\label{sec:fragestellung:anderebeziehungen}
	
	In der aktuellen \textit{STU}-Implementierung m�ssen andere Beziehungen manuell umgesetzt werden. Dies gilt
	auch f�r zirkul�re und reflexive, sowie alle nicht-bin�ren Beziehungen.



\section{Fortlaufendes Beispiel}
\label{sec:fragestellung:beispiel}

% Beispiel einleiten
Ein einheitliches fortlaufendes Beispiel soll der Arbeit als Grundlage dienen. Die Problemstellung besteht aus
einem Modell und einer Menge von Testdaten. Diese Testdaten dienen als Grundlage f�r die Diskussion der unterschiedlichen
Modellierungsvarianten.

	\subsection{Anforderungen an das Beispiel}
	\label{sec:fragestellung:beispiel:voraussetzungen}

	Der Schwerpunkt der Modellierung liegt bei der Darstellung von Beziehungstypen zwischen Entit�tstypen. Dabei soll die
	Problemstellung einerseits nicht zu komplex sein, damit sie �berschaubar bleibt. Andererseits soll sie komplex genug
	sein, um m�glichst alle Beziehungsarten zwischen Entit�ten abzudecken.
	Die Testdaten sollten gleichzeitig ein \textit{Standard Fixture} und ein \textit{Minimal Fixture} darstellen
	(\refsec{sec:grundlagen:konzepte:tests}).
	

	\subsection{Gew�hlte Problemstellung}
	\label{sec:fragestellung:beispiel:gewaehlte_problemstellung}
	Das gew�hlte Beispiel stellt eine starke Vereinfachung des Pr�fungswesens an Hochschulen dar. Auf eine praxisnahe
	Umsetzung wird zugunsten der Komplexit�t verzichtet. Personenbezogene Begriffe werden in der maskulinen Form verwendet,
	ohne dabei Aussagen �ber das Geschlecht der repr�sentierter Personen zu machen. Es beinhaltet die folgenden vier 
	Entit�tstypen:

	\begin{itemize}
		\item \textbf{Professor}: Ein Professor leitet Lehrveranstaltungen.
		\item \textbf{Lehrveranstaltung}: Eine Lehrveranstaltung wird von einem Professor geleitet. Es kann zu jeder
			Lehrveranstaltung eine Pr�fung geben.
		\item \textbf{Pr�fung}: Eine Pr�fung ist einer Lehrveranstaltung zugeordnet. Au�erdem hat mindestens ein Professor
			Aufsicht.
		\item \textbf{Student}: Studenten k�nnen an Lehrveranstaltungen und an Pr�fungen teilnehmen. Studenten haben au�erdem 
			die M�glichkeit, Tutoren von Lehrveranstaltungen zu sein.
		\item \textbf{Raum}: Ein Professor kann einen Raum als B�ro zugewiesen bekommen.
	\end{itemize}
	
	Die Beziehungen der Entit�tstypen stellen sich wie folgt dar: 
	\begin{itemize}
		\item \textbf{leitet}: Eine Lehrveranstaltung muss von genau einem Professor geleitet werden, ein Professor kann beliebig viele
		  oder keine Lehrveranstaltungen leiten.
		\item \textbf{gepr�ft}: Eine Pr�fung ist genau einer Lehrveranstaltung zugeordnet. Eine Lehrveranstaltung kann mehrere Pr�fungen 
		  haben (z.B. Nachschreibpr�fung).
		\item \textbf{beaufsichtigt}: Eine Pr�fung muss mindestens von einem Professor beaufsichtigt werden, ein Professor kann in 
		  beliebig vielen Pr�fungen Aufsicht haben. 
		\item \textbf{besucht}: Jeder Student kann beliebig vielen Lehrveranstaltungen besuchen. Lehrveranstaltungen ben�tigen jedoch 
		  mindestens drei Besucher um stattzufinden und sind aus Kapazit�tsgr�nden auf 100 Teilnehmer begrenzt.
		\item \textbf{ist Tutor}: Jeder Student kann bei beliebig vielen Lehrveranstaltungen Tutor sein und jede Lehrveranstaltung
			kann beliebig viele Tutoren haben. 
		\item \textbf{schreibt}: Jeder	Student kann an beliebig vielen Pr�fungen teilnehmen und umgekehrt eine Pr�fung von einer
		  beliebigen Anzahl von Studenten geschrieben werden.
		\item \textbf{hat B�ro}: Jeder Professor hat ein B�ro. Ein Raum kann einem oder keinem Professor zugeordnet sein.
	\end{itemize}

	Abbildung \ref{img:example_er} zeigt das Beispiel grafisch in Form eines ER-Diagramms. Den verschiedenen Entit�tstypen
	werden dabei Attribute zugeordnet. 
	
	\begin{figure}[H]
		\centering
		 \includegraphics[width=0.65\textwidth]{images/fragestellung/example_hochschule_er.pdf}
		\caption{ER-Diagramm des fortlaufenden Beispiels}\label{img:example_er}
	\end{figure}

	Das entsprechende relationale Datenbank-Schema wird in Abbildung \ref{img:example_relational} dargestellt. 
	Assoziative Tabellen realisieren die n:m-Beziehungen.
	
	\begin{figure}[H]
		\centering
		 \includegraphics[width=0.95\textwidth]{images/fragestellung/example_hochschule_relational.pdf}
		\caption{Relationales Datenbank-Schema des fortlaufenden Beispiels}\label{img:example_relational}
	\end{figure}

	Das Attribut "`fakultaet"' in der Tabelle Professor soll als Aufz�hlungstyp (enumeration) realisiert werden.
	M�gliche Werte sind: Architektur, Bauingenieurwesen, Elektrotechnik, Informatik, Maschinenbau und Wirtschaftswesen.
	Das Foto des Professors wird als BLOB dargestellt.
	\nomenclature{BLOB}{Binary Large Object)}
	

	\subsection{Beispiel-Use-Cases}
	\label{sec:fragestellung:beispiel:usecases}
	
	\todo{Use-Cases integrieren?}

	Um den einen Kompromiss f�r die Komplexit�t der Testdaten zu finden, werden vier Fragestellungen definiert. Diese
	Fragen sollen dabei helfen, den Umfang der Testdaten bestimmen zu k�nnen. Die Fragen stellen sich wie folgt dar:

	\begin{enumerate}
		\item Welcher Professor unterrichtet die meisten Studenten?
		\item Welcher Student nimmt an den meisten Pr�fungen teil?
		\item Welcher Student ist Tutor und nimmt gleichzeitig an der Pr�fung teil?
		\item Welcher Professor macht die wenigste Aufsicht in Fremdveranstaltungen (Lehrveranstaltungen eines anderen
			Professors)?
	\end{enumerate}


\todo{�berleitung von Use-Cases auf Modellierungsvarianten}


\section{Modellierungsvarianten der Testdaten f�r DbUnit}
\label{sec:fragestellung:modellierung}
	
	In \textit{DbUnit} werden die Datenbankzust�nde durch DataSets repr�sentiert. F�r einen Test werden gew�hnlich zwei
	DataSets ben�tigt: das erste f�r den Anfangszustand, das zweite f�r den erwarteten Zustand. Allerdings bieten
	DbUnit-DataSets nur begrenzte M�glichkeiten, das DataSet mit dem erwarteten Zustand aus dem DataSet mit dem
	Anfangszustand zu erzeugen.

	Im Folgenden werden verschiedene Modellierungsarten f�r DbUnit-DataSets diskutiert. Diese soll anhand der im
	n�chsten Abschnitt beschriebenen Kriterien erfolgen. Die Ergebnisse stellen die Grundlage
	f�r die konkretere Zielsetzung dar.

	\subsection{Kriterien f�r Bewertung}
	\label{sec:fragestellung:modellierung:sprachkriterien}
	
	F�r die Bewertung von Modellierungssprachen werden die folgenden Kriterieren verwendet. Einige Punkte
	sind messbar, andere wiederum relativ subjektiv. Als Vorbild f�r die beiden letzten Punkte dient die
	Norm ISO IEC 9126. 

	\begin{itemize}

		\item \textbf{Zeilen}: Die Anzahl der Zeilen, die f�r ein DataSet ben�tigt werden. 
		
		\item \textbf{Zeichen pro Zeile}: Ist die Sprache f�r die Darstellung auf Bildschirmen geeignet?
		
		\item \textbf{Typsicherheit}: Wann und wie werden falsche Datentypen bei der Modellierung erkannt?
		
		\item \textbf{Redundanz}: M�ssen Daten oder Sprachelemente redundant verwendet werden?

		\item \textbf{Benutzbarkeit (Verst�ndlichkeit und Erlernbarkeit)}: 
		  Wie gut dr�ckt die Sprache aus, welche Daten und Beziehungen modelliert werden? Wie leicht
			ist die Sprache zu lernen?
		
		\item \textbf{Modifizierbarkeit}: Wie leicht lassen sich Daten �ndern? Wie leicht k�nnen
		  k�nnen bestehende Daten an ein neues Datenbankschema angepasst werden?

	
	\end{itemize}


	\subsection{XML-DataSet}
	\label{sec:fragestellung:modellierung:xml}
	
	Eine M�glichkeit, ein DataSet f�r DbUnit zu modellieren, stellt XML dar. DbUnit selbst bietet zwei Varianten an, DataSets
	�ber XML zu modellieren.
	
	Die erste Variante stellt das \texttt{XmlDataSet} dar. Diese Klasse liest eine XML-Datei nach einem von DbUnit
	vorgegebenen Dokumententyp ein. Das Listing \ref{listing:xmldataset} zeigt einen Ausschnitt einer solchen XML-Datei,
	in dem die beiden Tabellen \textit{Professor} und \textit{Lehrveranstaltung} definiert werden.
	
	\lstSetXML
	\begin{lstlisting}[caption=XML-DataSet, label=listing:xmldataset]
<!DOCTYPE dataset SYSTEM "dataset.dtd">
<dataset>
    <table name="PROFESSOR">
        <column>id</column>
        <column>name</column>
        <column>vorname</column>
        <column>titel</column>
        <column>fakultaet</column>
        <row>
            <value>1</value>
            <value>W�sch</value>
            <value>J�rgen</value>
            <value>Prof. Dr.-Ing.</value>
            <value>Informatik</value>
        </row>
        <row>
            <value>2</value>
            <value>Haase</value>
            <value>Oliver</value>
            <value>Prof. Dr.</value>
            <value>Informatik</value>
        </row>
    </table>
    <table name="LEHRVERANSTALTUNG">
        <column>id</column>
        <column>professor_id</column>
        <column>name</column>
        <column>sws</column>
        <column>ects</column>
        <row>
            <value>1</value>
            <value>2</value>
            <value>Verteilte Systeme</value>
            <value>4</value>
            <value>5</value>
        </row>
        <row>
            <value>2</value>
            <value>2</value>
            <value>Design Patterns</value>
            <value>4</value>
            <value>3</value>
        </row>
    </table>
	...
</dataset>
	\end{lstlisting}
	
	Die Bewertung anhand der Kriterien stellt sich f�r das \texttt{XmlDataSet} wie folgt dar:
	\begin{itemize}

		\item \textbf{Zeilen}: Die XML-Datei mit den Beispiel-Daten umfasst 127 Zeilen. F�r jede Entit�t
		  werden jeweils eine Zeile pro Attribut und weitere zwei Zeilen f�r die umschlie�enden XML-Tags.
		
			DbUnit-konforme XML-Dateien wachsen schnell in vertikaler Richtung und enthalten unter Umst�nden
			auch viel syntaktischen Overhead. Von den rund 30 gezeigten Zeilen enthalten nur zehn Zeilen
			wirkliche Daten bzw. dr�cken Beziehungen aus (Zeilen 21 und 26).

		\item \textbf{Zeichen pro Zeile}: Mit maximal 47 Zeichen pro Zeile ist diese Modellierungsvariante
		  in Bezug auf die Breite gut f�r die Bildschirmdarstellung geeignet.
		
		\item \textbf{Typsicherheit}: 
			Zur Modellierung m�ssen Meta-Informationen zu den Daten hinterlegt werden. Diese beschr�nken sich allerdings auf
			die Bezeichnungen der Spalten (Zeilen 4-8 und 25-29). Da weitere Meta-Informationen fehlen, k�nnen fehlerhafte
			Datentypen oder Verst��e gegen Datenbank-Constraints erst zur Laufzeit beim Einspielen des DataSets erkannt werden.
		
		\item \textbf{Redundanz}: 
		  Das Modellieren von Beziehungen f�hrt zu Werte-Redundanz. Die konkreten Werte von Prim�rschl�sseln m�ssen an anderer
			Stelle als Fremdschl�ssel verwendet werden.

		\item \textbf{Benutzbarkeit (Verst�ndlichkeit und Erlernbarkeit)}: 
			Die positiven Eigenschaften bei der Modellierung mit XML sind unter anderem, dass f�r XML ein breites Angebot an
			Werkzeugen zur Verf�gung steht. Diese k�nnen �ber den Dokumententyp pr�fen, ob die Datei den Regeln entspricht.
			Der Umgang mit XML-Dateien kann als bekannt angenommen werden f�r die Zielgruppe.
			
			Die manuelle Pflege von Prim�r- und Fremdschl�sseln ist un�bersichtlich und damit fehleranf�llig. Die Value-Tags
			selbst lassen keinen R�ckschluss auf die Spalte zu, die sie repr�sentieren. Das erschwert die Lesbarkeit.
			
			In Bezug auf Verst�ndlichkeit zeigt die XML-Datei Schw�chen: Ohne zus�tzliche Kommentare ist eine solche XML-Datei 
			weder leicht zu lesen noch leicht zu pflegen. Auch Beziehungen sollten �ber Kommentare verdeutlicht werden.
		
			DbUnit unterst�tzt BLOBs in XML in Form Base64-codierter Daten. Bei gr��eren Datenmengen leidet die �bersicht unter dem
			Einbetten von BLOBs, nicht nur wegen der dem zus�tzlichen Platzbedarf aufgrund der Codierung. Spezielle Mechanismen,
			BLOBs aus anderen Dateien einzulesen, bringt DbUnit nicht mit. Solche Funktionen m�ssen manuell implementiert werden.
			
		\item \textbf{Modifizierbarkeit}: Daten lassen sich relativ leicht �ndern -- sofern man die richtige Stelle gefunden
		  hat, was ohne Kommentare nicht immer so leicht ist. Eine solche XML-Datei an ein neues Datenbank-Schema anzupassen
		  kann einfach aber auch m�hsam sein. Das Umbenennen von Spalten ist sehr einfach, das Entfernen oder Hinzuf�gen 
			von Spalten bei umfangreichen Daten ohne den geschickten Umgang mit Text-Editoren sehr umst�ndlich.
		  
	\end{itemize}
	
	
	Ein gro�er Nachteil bei der Nutzung von \texttt{XmlDataSet} ist, dass der erwartete Datenbankzustand selbst wieder den 
	kompletten Datenbankbestand umfassen muss. DbUnit erlaubt zwar mehrere DataSets zu einem zusammenzufassen, das Entfernen 
	von Datens�tzen ist dar�ber aber nicht m�glich. Mehrere XML-Dateien mit �hnlichen, �berwiegend sogar gleichen Daten,
	sorgen f�r ein hohes Ma� an Redundanz. Dar�ber hinaus sieht DbUnit keinen Mechanismus f�r die Komposition von XML-DataSets
	auf Modellierungsebene vor, d.h. es geht aus einer solchen XML-Datei nicht hervor, dass sie auf anderen DataSets
	aufbaut und diese erweitert.
	
	
	% Modellieren Assoziativer Tabellen ansprechen?

  Das \texttt{FlatXmlDataSet} stellt die zweite Variante dar. Hierbei gibt es keine
	von DbUnit vorgegebene DTD, da die Tags den Tabellen-Namen entsprechen\footnote{Es ist m�glich, eine eigene DTD zu
	definieren.}. Eine solche XML-Datei kommt ohne explizite Meta-Informationen zu den Tabellen aus. Stattdessen stellen sie
	eine Art Sprachelement dar und werden f�r die Zuweisung der Werte verwendet. In Bezug auf die Meta-Informationen
	ist das \texttt{FlatXmlDataSet} �bersichtlicher als das XmlDataSet (\reflst{listing:flatxmldataset}).
  
	\lstSetXML
	\begin{lstlisting}[caption=Flat-XML-DataSet, label=listing:flatxmldataset]
<?xml version='1.0' encoding='UTF-8'?>
<dataset>
    <PROFESSOR id="1" 
        name="W�sch"
        vorname="J�rgen"
        titel="Prof. Dr.-Ing."
        fakultaet="Informatik" />
    <PROFESSOR id="2" 
        name="Haase"
        vorname="Oliver"
        titel="Prof. Dr."
        fakultaet="Informatik" />
    <LEHRVERANSTALTUNG id="1"
        professor_id="2"
        name="Verteilte Systeme"
        sws="4"
        ects="5" />
    <LEHRVERANSTALTUNG id="2"
        professor_id="2"
        name="Design Patterns"
        sws="4"
        ects="3" />
...
</dataset>
	\end{lstlisting}

	Das \texttt{FlatXmlDataSet} hat gro�e �hnlichkeit zum \texttt{XmlDataSet}, das zeigt sich auch in
	der Bewertung. Einige vorher genannte Punkte gelten hier weiterhin.
	
	\begin{itemize}

		\item \textbf{Zeilen}: Die selben Beispieldaten lassen sich hier mit 63 Zeilen ausdr�cken. Die Datei
		  kommt mit weniger Meta-Informationen und etwas weniger syntaktischen Overhead aus. Allerdings
		  sollte auch hier jedes Attribut in eine Zeile geschrieben werden.
		
		\item \textbf{Zeichen pro Zeile}: Mit maximal 40 Zeichen pro Zeile f�r die gew�hlten Testdaten ist
		  das \texttt{FlatXmlDataSet} f�r die Bildschirmdarstellung gut geeignet.
		
		\item \textbf{Typsicherheit}: Wie auch beim \texttt{XmlDataSet} k�nnen die Typen erst beim
		  Einspielen in die Datenbank �berpr�ft werden.
		  
		\item \textbf{Redundanz}: Das Beschreiben von Beziehungen erfordert die selbe Daten-Redundanz wie
		  beim \texttt{XmlDataSet}.

		\item \textbf{Benutzbarkeit (Verst�ndlichkeit und Erlernbarkeit)}:
		  Durch die fehlende Hierarchie wirkt das \texttt{FlatXmlDataSet} etwas un�bersichtlich. Die
			Spalten-Bezeichner stellen eine Art Sprachelement dar, d.h. sie werden als XML-Attribut-Bezeichner
			bei der Datenzuweisung verwendet. Das ist �bersichtlich und verst�ndlich. Au�erdem m�ssen
			die Attribute nicht zwingend in der selben Reihenfolge angegeben werden.
			
		\item \textbf{Modifizierbarkeit}:
		  Das �ndern der Daten wird dadurch erleichtert, dass Spaltennamen und Wert direkt beieinander stehen.
			Ansonsten gelten die bereits f�r das \texttt{XmlDataSet} genannten Punkte.
	
	\end{itemize}
	


	\subsection{Default-DataSet}
	\label{sec:fragestellung:modellierung:java}
	
	DbUnit erlaubt auch die programmatische Modellierung von DataSets. Dazu stellt es die Klasse \texttt{DefaultDataSet}
	bereit. Mit den Mitteln, die eine Programmiersprache wie Java bietet, lassen sich einige der Nachteile in Verbindung
  mit den XML-basierten DataSets direkt umgehen.
	
	So k�nnen Beziehungen mit Hilfe symbolischer Konstanten ausdrucksst�rker modelliert werden. Auch wenn die Beziehungen
	immer noch etwas umst�ndlich modelliert werden m�ssen, k�nnen symbolische Konstanten dabei helfen, Redundanz zu vermeiden
	und damit das Risiko f�r Fehler zu senken.

	\lstSetJava
	\begin{lstlisting}[caption=Default-DataSet, label=listing:javadataset]
DefaultTable professor = new DefaultTable(
		"professor",
		new Column[] { 
			new Column("id", DataType.INTEGER),
			new Column("name", DataType.VARCHAR), 
			new Column("vorname", DataType.VARCHAR), 
			new Column("titel", DataType.VARCHAR), 
			new Column("fakultaet", DataType.VARCHAR), 
		}
	);
professor.addRow(new Object[] { 
			Parameters.Professor.WAESCH_ID,
			"W�sch",
			"J�rgen",
			"Prof. Dr.-Ing.",
			"Informatik",
		});
professor.addRow(new Object[] { 
			Parameters.Professor.HAASE_ID,
			"Haase",
			"Oliver",
			"Prof. Dr.",
			"Informatik",
		});
dataSet.addTable(professor);

DefaultTable lehrveranstaltung = new DefaultTable(
		"lehrveranstaltung", 
		new Column[] {
			new Column("id", DataType.INTEGER),
			new Column("professor_id", DataType.INTEGER),
			new Column("name", DataType.VARCHAR), 
			new Column("sws", DataType.INTEGER),
			new Column("ects", DataType.INTEGER),
		}
	);
lehrveranstaltung.addRow(new Object[] {
			Parameters.Lehrveranstaltung.VSYSTEME_ID,
			Parameters.Professor.HAASE_ID, 
			"Verteilte Systeme",
			4,
			5,
		});
lehrveranstaltung.addRow(new Object[] {
			Parameters.Lehrveranstaltung.DESIGN_PATTERNS_ID,
			Parameters.Professor.HAASE_ID,
			"Design Patterns",
			4,
			3,
		});
dataSet.addTable(lehrveranstaltung);
  \end{lstlisting}
	
	Diese Variante l�st allerdings nicht alle Probleme: So m�ssen immer noch Meta-Informationen zu den Tabellen
	modelliert werden (Zeilen 3-9 und 29-36). Obwohl diese sogar Typinformationen beinhalten, werden Typ-Fehler erst
	zur Laufzeit beim Einspielen in die Datenbank erkannt. Der Einsatz von symbolischen 
	Konstanten erleichtert zwar die Pflege des DataSets, dennoch lassen sich Konstanten doppelt belegen oder auch
	Prim�rschl�ssel einer falschen Datenbank als Fremdschl�ssel angegeben werden.
	
	�hnlich wie f�r die Modellierung �ber XML-Dateien sind f�r eine �bersichtliche Formatierung  viele Zeilen notwendig
	und umfangreiche Datensets werden daher un�bersichtlich. Insgesamt bietet die Nutzung dieser Java-DataSets 
	wenig Vorteile gegen�ber den XML-DataSets.
	
  \todo{Sprache gem�� Kriterien bewerten}	
	\begin{itemize}

		\item \textbf{Zeilen}:
		
		\item \textbf{Zeichen pro Zeile}: 
		
		\item \textbf{Typsicherheit}: 
		
		\item \textbf{Redundanz}: 

		\item \textbf{Benutzbarkeit (Verst�ndlichkeit und Erlernbarkeit)}: 
	
		\item \textbf{Modifizierbarkeit}: 

	\end{itemize}

	\subsection{STU-DataSet}
	\label{sec:fragestellung:modellierung:sbtesting}
	
	Die Bibliothek \textit{STU} erm�glicht die Modellierung von DbUnit-DataSets mit Hilfe eines
	Datenbank-Modell-spezifischen API. Dieses API wird �ber einen Generator erzeugt (siehe auch
	\ref{sec:grundlagen:stu}). 
	
	\textit{STU} f�hrt eine eigene DataSet-Klasse ein,
	�ber die die Daten modelliert werden. Diese DataSet-Klasse kann bei Bedarf von den aktuellen Daten ein
	DbUnit-DataSet erzeugen. Auf diese Weise k�nnen DataSets aus \textit{STU} einfacher und umfangreicher
	als DbUnit-DataSets modifiziert werden, wie z.B. das L�schen von Zeilen.
	
	Auf diese Weise k�nnen mit \textit{STU} verh�ltnism��ig einfach Varianten eines DbUnit-DataSets
	erzeugt werden, z.B. ein DataSet mit dem Ausgangszustand und ein DataSet mit dem erwarten Zustand am Ende des Tests.
	
	Die Java-DSL sorgt f�r statische Typsicherheit, so dass Java-IDEs fehlerhafte Typen bereits w�hrend der
	Entwicklung kenntlich machen. Verglichen mit den DbUnit-Xml-DataSets und dem Default-DataSet
	ist die Syntax ist etwas kompakter und ausdrucksst�rker. Spaltennamen und Werte stehen beieinander und nicht 
	�ber die Datei verteilt.

	\lstSetJava
	\begin{lstlisting}[caption=STU DataSet (1), label=listing:sbtestingdataset_old]
table_Professor
	.insertRow()
		.setId(Parameters.Professor.HAASE_ID)
		.setName("Haase")
		.setVorname("Oliver")
		.setTitel("Prof. Dr.")
		.setFakultaet("Informatik")
	.insertRow()
		.setId(Parameters.Professor.WAESCH_ID)
		.setName("W�sch")
		.setVorname("J�rgen")
		.setTitel("Prof. Dr.-Ing.")
		.setFakultaet("Informatik");

table_Lehrveranstaltung
	.insertRow()
		.setId(Parameters.Lehrveranstaltung.VSYSTEME_ID)
		.setProfessorId(Parameters.Professor.HAASE_ID)
		.setName("Verteilte Systeme")
		.setSws(4)
		.setEcts(5)
	.insertRow()
		.setId(Parameters.Lehrveranstaltung.DESIGN_PATTERNS_ID)
		.setProfessorId(Parameters.Professor.HAASE_ID)
		.setName("Design Patterns")
		.setSws(4)
		.setEcts(3);
	\end{lstlisting}

	Die Modellierung von Beziehungen stellt sich als �hnlich problematisch wie bei den bisherigen Java-DataSets dar
	(\refsec{sec:fragestellung:modellierung:java}). Nach wie vor w�chst das DataSet vertikal in der Datei. 
	
	Eine Erweiterung des Datenbank-Modells und des Generators kann die Modellierung von Beziehungen bereits etwas
	verbessern. Diese Erweiterung erlaubt es, anstelle eines Fremdschl�ssels eine vorher eingef�gte Zeile 
	anzugeben (\reflst{listing:sbtestingdataset}, Zeilen 20 und 27). Hier k�nnen referenzierte Prim�rschl�ssel auch
	automatisch vergeben werden.

  \todo{Sprache gem�� Kriterien bewerten}	
	\begin{itemize}

		\item \textbf{Zeilen}:
		
		\item \textbf{Zeichen pro Zeile}: 
		
		\item \textbf{Typsicherheit}: 
		
		\item \textbf{Redundanz}: 

		\item \textbf{Benutzbarkeit (Verst�ndlichkeit und Erlernbarkeit)}: 
	
		\item \textbf{Modifizierbarkeit}: 

	
	\end{itemize}

	\lstSetJava
	\begin{lstlisting}[caption=STU DataSet (2), label=listing:sbtestingdataset]
RowBuilder_Professor haase = 
	table_Professor
		.insertRow()
			.setName("Haase")
			.setVorname("Oliver")
			.setTitel("Prof. Dr.")
			.setFakultaet("Informatik");
RowBuilder_Professor waesch = 
	table_Professor
		.insertRow()
			.setName("W�sch")
			.setVorname("J�rgen")
			.setTitel("Prof. Dr.-Ing.")
			.setFakultaet("Informatik");

RowBuilder_Lehrveranstaltung vsys = 
	table_Lehrveranstaltung
		.insertRow()
			.setName("Verteilte Systeme")
			.refProfessorId(haase)
			.setSws(4)
			.setEcts(5);
RowBuilder_Lehrveranstaltung design_patterns = 
	table_Lehrveranstaltung
		.insertRow()
			.setName("Design Patterns")
			.refProfessorId(haase)
			.setSws(4)
			.setEcts(3);
	\end{lstlisting}
	

	

  \chapter{Modellierung der Test-Daten}
\label{chap:modellierung}



\section{DSL-Entw�rfe}

	\subsection{Entwurf 1}
	
	Eine DSL, die sich stark an \textit{SB Testing DB} orientiert, k�nnte wie folgt aussehen:
	
	\begin{lstlisting}[caption=M�gliche DSL (1), label=listing:dslentwurf1]
HAASE = professor {
	name			"Haase"
	vorname   "Oliver"
	titel     "Prof. Dr."
  fakultaet "Informatik"
}

WAESCH = professor {
	name			"W�sch"
	vorname   "J�rgen"
	titel     "Prof. Dr.-Ing."
  fakultaet "Informatik"
}
	
VSYS = lehrveranstaltung {
	name			"Verteilte Systeme"
  sws       4
	ects      5
}
	
DPATTERNS = lehrveranstaltung {
	name 			"Design Patterns"
	sws       4
	ects      3
}

...

HAASE leitet VSYS
HAASE leitet DPATTERNS
HAASE beaufsichtigt	P_DPATTERNS
WAESCH beaufsichtigt P_VSYS
...

	\end{lstlisting}
	
	Diese DSL kommt ohne manuell vergebene ID-Nummern aus und verwendet Variablennamen f�r die Modellierung von Beziehungen. 
	Da f�r jeden Wert eine eigene Zeile verwendet wird, werden umfangreiche Daten schnell un�bersichtlich. Die Beschreibung
	der Beziehungen abseits der Definition der Daten erschwert den Umgang mit den Daten und die �bersicht ebenfalls.


	\subsection{Entwurf 2}
	
	Ein leicht abgewandelter Entwurf zeigt, wie sich die Beziehungen n�her an den eigentlichen Daten beschreiben lassen k�nnten.
	An dem Problem, dass die Daten relativ schnell in vertikaler Richtung wachsen, �ndert das jedoch nichts.
	

	\begin{lstlisting}[caption=M�gliche DSL (2), label=listing:dslentwurf2]
HAASE = professor {
	name      "Haase"
	vorname   "Oliver"
	titel     "Prof. Dr."
  fakultaet "Informatik"
	leitet    VSYS, DPATTERNS
	beaufsichtigt	P_DPATTERNS
}

WAESCH = professor {
	name      "W�sch"
	vorname   "J�rgen"
	titel     "Prof. Dr.-Ing."
  fakultaet "Informatik"
	beaufsichtigt	P_VSYS
}
	
VSYS = lehrveranstaltung {
	name			"Verteilte Systeme"
  sws       4
	ects      5
}
	
DPATTERNS = lehrveranstaltung {
	name 			"Design Patterns"
  sws       4
	ects      3
}

...
	\end{lstlisting}
	

	\subsection{Entwurf 3}
	
	Der dritte Entwurf versucht die Daten durch eine tabellarische Struktur �bersichtlich zu gestalten. Sie kommt mit
	wenig syntaktischem Ballast aus. Ein Label vor einer Tabelle dr�ckt aus, welche Daten folgen (Zeilen 1 und 6). Die
	Tabelle selbst beginnt mit einer Kopfzeile, die die Spaltenreihenfolge beschreibt (Zeilen 2 und 7).

	\lstSetTiny
	\begin{lstlisting}[caption=M�gliche DSL (3), label=listing:dslentwurf3]
professor:
REF    || name    | vorname  | titel            | fakultaet    | leitet          | beaufsichtigt
HAASE  || "Haase" | "Oliver" | "Prof. Dr."      | "Informatik" | VSYS, DPATTERNS | P_DPATTERNS   
WAESCH || "W�sch" | "J�rgen" | "Prof. Dr.-Ing." | "Informatik" |                 | P_VSYS
	
lehrveranstaltung:
REF       || name                | sws | ects
VSYS      || "Verteilte Systeme" | 4   | 5
DPATTERNS || "Design Patterns"   | 4   | 3

...
	\end{lstlisting}
	\lstSetNotmal
	
	Der Entwurf sieht vor, dass Beziehungen innerhalb beider Entit�tstypen ausgedr�ckt werden k�nnen. So kann
	eine Tabelle um Spalten f�r Beziehungen erg�nzt werden, die in dieser Form nicht Teil des relationalen
	Modells (\refimg{img:example_relational}) sind. Dazu geh�ren die Spalten "`leitet"' und "`beaufsichtigt"'
	der Professor-Tabelle. Erstere dr�ckt die 1:n-Beziehung zu einer Lehrveranstaltung aus, letztere die
	m:n-Beziehung zu Pr�fungen.
	
	Probleme bzw. Nachteile in der Darstellung k�nnen auftreten, wenn die L�nge der Werte in einer Spalte stark
	variiert. Die Spaltenbreite wird vom l�ngsten Element bestimmt. Der Entwickler ist selbst daf�r verantwortlich,
	die �bersichtliche Darstellung einzuhalten. Auf Tabulatoren sollte unter  Umst�nden verzichtet werden, da sie von
	verschiedenen Editoren unterschiedlich dargestellt werden k�nnen. Bei vielen Spalten w�chst diese Darstellung
	horizontal. Bei optionalen Spalten bzw. kaum genutzte Spalten kann die tabellarische Darstellung un�bersichtlich
	werden.
	
	Einige Entwicklungsumgebungen wie Eclipse bieten spezielle Block-Bearbeitungsfunktionen an, die beim Arbeiten an
	einer Tabellen-DSL hilfreich sein kann. So k�nnen beispielsweise in einer Spalte �ber mehrere Zeilen hinweg 
	Leerzeichen eingef�gt oder entfernt werden.
	
	Zur besseren �bersicht kann es bei gr��eren Tabellen sinnvoll sein, den Tabellenkopf zu wiederholen.
	
	Der Double-Pipe-Operator (||) soll die Spalte mit dem Entit�tsidentifikatoren visuell von den Datenspalten  
	trennen.
	
	
\section{Wahl der DSL}

Der dritte Entwurf zeigt, dass eine tabellarische Schreibweise viele Schw�chen der anderen Varianten ausmerzt.
Die Darstellung wirkt �bersichtlich, da Tabellen ... \todo{Hier w�re eine Quelle super, dass Menschen vertraut
mit Tabellen sind}

	

\section{Implementierungsvorbereitung}
\label{sec:modellierung:wahlimplementierung}

Da sich die DSL in die bisherige Werkzeug-Kette von Seitenbau integrieren lassen soll
(\refsec{sec:anforderungen:allgemeineanforderungen}), sollte die DSL in Java nutzbar sein. Zwar kann eine DSL
grunds�tzlich auch in Java realisiert werden, doch die M�glichkeiten diesbez�glich sind relativ eingeschr�nkt 
und die DSL sieht immer noch nach Java aus. Es lassen sich allerdings auch andere Sprachen im Java-Umfeld nutzen.
Eine davon ist \textit{Groovy}. Groovy ist eine dynamisch typisierte Sprache\footnote{Im Gegensatz zu statisch 
typisierten Sprachen finden bei dynamisch typisierten Typ-�berpr�fungen �berwiegend zur Laufzeit statt.}, die
direkt in Java-Bytecode �bersetzt wird und damit auch in einer Java Virtual Machine ausgef�hrt wird. Sie teilt
sich das Objekt-Modell mit Java, so dass aus Groovy heraus instantiierte Objekte auch in der Host-Anwendung 
nutzbar sind (und umgekehrt). Auch wenn Java-Code bis auf wenige Ausnahmen g�ltiger Groovy-Code und sich dort
gleich verh�lt, enth�lt Groovy Techniken, die den Code mehr wie eine nat�rliche Sprache aussehen lassen.
So kann oftmals auf die Semikolons am Ende einer Anweisung verzichtet werden, und auch auf das Einklammern
von Parametern kann bei Methoden aufrufen verzichtet werden (wenn die Methode genau einen Parameter erwartet).
Au�erdem kann statt dem Punkt zwischen Objekt und Methode beim Aufruf verzichtet werden.

Listing \ref{listing:groovyexamples} zeigt einen Befehl einmal in typischer Java-Syntax und einmal mit den
Syntax-Vereinfachungen von Groovy:

	\begin{lstlisting}[caption=Vereinfachung von Ausdr�cken in Groovy, label=listing:groovyexamples]
myList.append("value 1").append("value 2");
myList append "value 1"  append "value 2"  
	\end{lstlisting}

Groovy hebt sich ferner durch die M�glichkeit Operatoren zu �berladen und durch Closures (Funktionsabschl�sse) von
Java ab. Ein Closure ist ein Codeblock, der wie eine Funktion aufgerufen und genutzt werden kann. In Java lassen
sich CLosures mit syntaktisch umfangreicheren Methoden-Objekten nachbilden. Ein Methoden-Objekt stellt eine
Instanz einer (m�glicherweise anonymen) Klasse dar, die nur eine Methode implementiert. \cite[40]{GROOVY_IM_EINSATZ} 
\todo{Quelle Kent Beck Smalltalk Best Practice Patterns} 
Die Unterst�tzung zur Meta-Programmierung stellt sich beim Implementieren einer DSL ebenfalls als n�tzlich
heraus. Dadurch ist es z.B. m�glich, abgeschlossene Klassen innerhalb von Groovy um Methoden zu erweitern oder auf
den Zugriff von nicht definierten Klassenelementen zu reagieren.

Aus diesen Gr�nden empfiehlt Ghosh in \cite[148]{DSLS_IN_ACTION} Groovy als Host f�r DSLs in Verbindung
mit Java-Anwendungen. 

	\subsection{Implementierungsvarianten}
	\label{sec:modellierung:implementierung:varianten}
	
	Eine DSL kann auf unterschiedliche Arten implementiert werden. Groovy bietet daf�r zwei M�glichkeiten der
	Meta-Programmierung an: Laufzeit-Meta-Programmierung und Compiler-Zeit-Meta-Programmierung, letzteres in Form von
	AST-Transformationen. Beide Ans�tze bieten individuelle Vorteile, die im folgenden diskutiert werden.  
	\nomenclature{AST}{Abstract Syntax Tree}


		\subsubsection{Laufzeit-Meta-Programmierung}
		
		Eine M�glichkeit, die DSL mit Hilfe von Laufzeit-Meta-Programmierung zu implementieren sieht eine 
		Klasse zum Parsen von Closures vor, die eine Tabelle beinhalten. Diese Klasse, \textit{TableParser},
		enth�lt daf�r die Methode \texttt{parseTableClosure}. Die Methode soll als Ergebnis eine Liste
		von Tabellenzeilen zur�ckliefern. Da an dieser Stelle noch keinerlei Interpretation der Tabellenwerte
		durchgef�hrt wird, stellt eine Tabellenzeile selbst ebenfalls eine Liste dar - aus den Objekten
		der Spalten.
		
		Der Ansatz ist, Operator-�berladen f�r das Parsen zu verwenden. Soll ein bin�rer Operator\footnote{Bin�r
		bezogen auf die Verkn�pfung zweier Werte und nicht auf das Zahlensystem} implementiert werden, ist die
		�bliche Vorgehensweise in Groovy, die Klasse des linken Operanden um eine entsprechende Methode f�r den
		Operator zu erweitern. Diese Methode tr�gt einen vorgegebenen Namen und erwartet als bin�rer Operator 
		den rechten Operanden als Parameter (eine  �bersicht findet sich beispielsweise in 
		\cite[58]{GROOVY_IM_EINSATZ}).
		
		Auch wenn sich dank der M�glichkeiten der Meta-Programmierung Klassen in Groovy zur Laufzeit um Methoden
		erg�nzen lassen, ist dieses Vorgehen nicht empfehlenswert f�r um eine Tabelle zu parsen. Dieser wenig
		generische Ansatz m�sste jeden in den Tabellen m�gliche Datentyp ber�cksichtigen - kommen neue Datentypen
		hinzu, m�sste der Code erweitert werden.
		\todo{M�gliche ungewollte Seiteneffekte}
		
		Groovy bietet allerdings auch eine zweite M�glichkeit f�r das Operator-�berladen an. Anstatt den Operator
		als Methode dem linken Operand (bzw. der Klasse) hinzuzuf�gen, wird er als statische Methode (in einer
		beliebigen Klasse) realisiert. Da eine statische Methode ohne Kontext ausgef�hrt wird, ben�tigt sie alle
		beteiligten Operanden als Parameter. Eine solche Methode wird als Kategoriemethode bezeichnet. 
		�ber das Schl�sselwort \textit{use}\footnote{\textit{use} wird in der Literatur meistens als Schl�sselwort
		bezeichnet, tats�chlich handelt es sich jedoch um eine Groovy-Methode in \texttt{java.lang.Object}}
		k�nnen die Kategoriemethoden in einem Closure verwendet werden. \cite[192]{GROOVY_IM_EINSATZ} 
		
		Listing \ref{listing:opoverloading.tableparser.base} zeigt das Grundger�st des Tabellenparsers:
		
		\begin{lstlisting}[caption=Tabellen-Parser Grundger�st mit Operator-�berladen, label=listing:opoverloading.tableparser.base]
class TableParser {
  
  static or(self, arg) {
		// ...
  }

  def parseTableClosure(Closure tableData){
    use(TableParser) {
      tableData()
    }
  }

}
		\end{lstlisting}
		
		Die Methode \texttt{or} erwartet zwei Parameter vom Typ \textit{Object}. Obwohl in Groovy alle Typen von
		\textit{Object} abgeleitet sind, gibt es Oder-Ausdr�cke, bei denen diese Methode nicht aufgerufen wird.
		Ein in der Klasse definierter Operator mit passenden Datentypen wird dieser allgemeinen Methode bevorzugt,
		z.B. bei zwei \textit{Integer}-Werten. Doch auch solche Operationen lassen sich �berschreiben, wenn f�r
		die Datentypen passende Kategoriemethoden definiert werden.

		Der Parser in der Form kann noch nicht mit selbst definierten Variablennamen f�r die Referenzenabbildung
		umgehen. Aus diesem Grund wird eine Methode \texttt{getProperty} definiert, die f�r jeden Variablennamen
		in der Tabelle aufgerufen werden soll. Dazu muss der Ausf�hrungskontext des Closures auf die Instanz
		des Tabellenparsers ge�ndert werden. Die �nderungen sind in Listing \ref{listing:opoverloading.tableparser.extended}
		dargestellt.

		\begin{lstlisting}[caption=Tabellen-Parser Grundger�st mit Operator-�berladen, label=listing:opoverloading.tableparser.extended]
class TableParser {
  
  static or(self, arg) {
		// ...
  }
	
  static or(Integer self, Integer arg) {
		// ...
  }

  static or(Boolean self, Boolean arg) {
		// ...
	}
	
	def getProperty(String property) {
		// ...
  }
	

  def parseTableClosure(Closure tableData){
    use(TableParser) {
      tableData.delegate = this		// Change closure's context
      tableData.resolveStrategy = Closure.DELEGATE_FIRST
      tableData()
    }
  }

}
		\end{lstlisting}
		
		Die statischen Methoden haben keinen Zugriff auf Instanz-Variablen der Klasse \textit{TableParser}. Ihre Ergebnisse
		k�nnen sie demnach auch nur in statische Elementen aufbewahren. Um die Klasse Thread-sicher zu machen, d.h. das
		gleichzeitige Parsen von Tabellen aus verschiedenen Threads heraus, wird f�r die Ergebnisse eine threadlokale
		Liste verwendet. \todo{thread local erkl�ren mit quelle} \cite{JAVA_CONCURRENCY_IN_PRACTICE}

				
		Die Laufzeit-Meta-Programmierung kann die Syntax der Sprache nicht beliebig erweitern. Groovy kennt keinen
		Double-Pipe-Operator. Deshalb kann dieser weder �berladen noch �ber Laufzeit-Meta-Programmierung eingef�hrt
		werden. Folglich ist es nicht m�glich, den dritten Entwurf �ber reine Laufzeit-Meta-Programmierung zu
		realisieren. Allerdings kann eine Syntax erreicht werden, die dem Entwurf sehr nahe kommt
		(\reflst{listing:dslentwurf3laufzeit}). Ein Platzhalter (Unterstrich) verhindert Syntax-Fehler, wenn in
		einer Spalte kein Wert vorkommt (siehe Zeile 4, Spalte "`leitet"'). Der Platzhalter k�nnte auch verwendet
		werden, um einem Datensatz keinen Bezeichner f�r Referenzen zu zu weisen. Aus Sicht des Parsers stellt
		der Unterstrich eine Variable dar.
		
		\lstSetTiny
		\begin{lstlisting}[caption=DSL-Entwurf 3 f�r Laufzeit-Meta-Programmierung angepasst, label=listing:dslentwurf3laufzeit]
def fixture = [
  professor: {
	  REF    | name    | vorname  | titel            | fakultaet    | leitet           | beaufsichtigt
		WAESCH | "W�sch" | "J�rgen" | "Prof. Dr.-Ing." | "Informatik" | _                | P_VSYS
		HAASE  | "Haase" | "Oliver" | "Prof. Dr."      | "Informatik" | VSYS & DPATTERNS | P_DPATTERNS
  },

  lehrveranstaltung: {
    REF       | name                | sws | ects
    VSYS      | "Verteilte Systeme" | 4   | 5    
    DPATTERNS | "Design Patterns"   | 4   | 3    
  },
		
  ...
]		
		\end{lstlisting}
		\lstSetNotmal
		
		

		\subsubsection{AST-Transformation}
		
		Die AST-Transformationen stellen ein m�chtiges Werkzeug zur Erweiterung der Syntax der Sprache dar. Mit Hilfe
		der Transformationen ist es m�glich, �nderungen am AST durchzuf�hren, bevor er in Java-Bytecode �bersetzt wird.
		
		Dass AST-Transformationen mehr syntaktische M�glichkeiten bieten, zeigt sich auch daran, dass hier der 
		Double-Pipe-Operator verwendet werden kann. Au�erdem k�nnen Labels erkannt werden und Daten einer Tabelle
		m�ssen nicht zwangsl�ufig in einem eigenen Block definiert werden.
		
		Allerdings muss zum Auswerten einer Tabelle bei AST-Transofrmationen ein relativ gro�er Aufwand betrieben werden.
		Der Zugriff auf den AST erfolgt dabei �ber das Visitor-Pattern
		(\cite[331ff]{DESIGN_PATTERNS}).
		
	\subsection{Implementierungsentscheidung}
	\label{sec:implementierung:entscheidung}
	
	Der Vergleich zwischen Laufzeit-Meta-Programmierung und AST-Transformation zeigt, dass sich Groovy als Host-Sprache
	f�r die DSL eignet. Die Laufzeit-Meta-Programmierung erlaubt zwar weniger Anpassungen an die Sprache, ist aber f�r
	die gew�nschte DSL ausreichend und die Umsetzung einfacher. 
		

\section{Realisisierung}
\label{sec:modellierung:realisierung}

		
\todo{"`Muster"' f�r 1:1, 1:n und m:n}

	\chapter{Generieren von Testdaten}
\label{chap:generieren}

Es gibt verschiedene Ans�tze zur Generierung von Testdaten f�r Datenbank-basierte Anwendungen:
\begin{enumerate}
  \item \textbf{Modell-basierte, Abfrage-unabh�ngige Generierung}:
    Anhand eines Datenbank-Modells werden Entit�ten mit Zufallswerten f�r die Attribute (Spalten) erzeugt. Es gibt
    einige kommerzielle Werkzeuge aber auch frei nutzbare Internet-Seiten f�r die Generierung. 
  
  \item \textbf{Modell-basierte, Abfrage-basierte Generierung}:
    Ausgehend von konkreten Abfragen (z.B. in SQL) werden f�r die Abfrage passende Daten erzeugt. Binnig beschreibt in
    \cite{DBLP:conf:icde:BinnigKL07} einen Ansatz, bei dem SQL-Abfragen als Grundlage f�r die Daten-Generierung verwendet
    werden. Mit AGENDA wird in \cite{Chays04anagenda} ein Toolset vorgestellt, das neben dem Datenbank-Schema den
    Anwendungsquellcode betrachtet.

  \item \textbf{Anonymisierung realer Daten}:
    Bei diesem Ansatz findet keine echte Generierung statt. Stattdessen werden Daten einer realen Anwendung 
    anonymisiert und f�r Tests verwendet.
  
\end{enumerate}

F�r die Generierung eines Standard Fixtures scheint nur die erste Variante sinnvoll zu sein: Es liegt bereits
ein Modell vor und die generierten Daten sollen idealerweise f�r alle Tests verwendet werden k�nnen.
Konkrete Anfragen als Grundlage f�r die Generierung sind nicht sinnvoll. Einerseits k�nnen sie
vom SUT verborgen werden k�nnen, andererseits eigenen sie eher f�r Fixtures, die f�r einen einzelnen Unit-Test
geeignet sind. Die Anonymisierung realer Daten stellt keine Daten-Generierung im eigentlichen Sinn dar und
setzt bestehende Daten voraus.

Eine Auswahl existierender Modell-basierter, Abfrage-unabh�ngiger Datengeneratoren wird im folgenden Abschnitt
auf die Anwendbarkeit hin untersucht.


\section{Betrachtung existierender Werkzeuge}
\label{sec:generieren:analyse}

Es gibt bereits eine Reihe von Werkzeugen zur Generierung von Zufallsdaten f�r Datenbanken. In wie weit sich diese
f�r die Aufgabenstellung nutzen lassen, soll im folgenden kurz analysiert werden.

  \subsection{Kommerzielle Werkzeuge}
  \label{sec:generieren:analyse:kommerziellewerkzeuge}
  Zu den betrachteten kommerziellen Anwendungen z�hlen:

  \begin{itemize}
    \item \textbf{Datanamic Data Generator MultiDB}: \\
      \url{http://www.datanamic.com/datagenerator/}
    \item \textbf{DTM Data Generator}: \\
      \url{http://www.sqledit.com/dg/}
    \item \textbf{forSQL Data Generator}: \\
      \url{http://www.forsql.com/}
    \item \textbf{Red Gate SQL Data Generator}: \\
      \url{http://www.red-gate.com/products/sql-development/sql-data-generator/}
  \end{itemize}

  Insgesamt sind die M�glichkeiten der Anwendungen relativ �hnlich. Die gr��ten Unterschiede aus Nutzer-Sicht liegen in der
  Bedienung. Die Werkzeuge arbeiten zufallsbasiert aber deterministisch, d.h. sie erzeugen bei gleichem Modell die gleichen
  Daten. Das sogenannte \textit{Seed}, mit dem der Zufallszahlengenerator f�r eine einzelne Spalte initialisiert wird, l�sst
  sich z.B. beim Red Gate SQL Data Generator komfortabel festlegen.

  Die Werkzeuge sind vor allem f�r die Generierung von gro�en Datenmengen (Massen-Daten) vorgesehen. Dies zeigt sich auch darin,
  dass sie Beziehungen auch nur zuf�llig modellieren. �ber eine entsprechend gro�e Menge an Testdaten soll dann auch jeder
  notwendige Fall abgedeckt sein. Die Menge der zu erzeugenden Testdaten l�sst sich f�r jede einzeln Tabelle konfigurieren.
  Eigene Vorschl�ge, wie viele Daten generiert werden sollten, machen die Werkzeuge nicht.

  \subsection{Andere Ans�tze}
  \label{sec:generieren:analyse:andereansaetze}
  
  In \cite{Houkjaer:2006:SRD:1182635.1164254} wird ein Algorithmus zur Generierung von Test-Daten vorgestellt, der
  das Datenbank-Modell als Graphen betrachtet. Tabellen stellen Knoten und ihre Beziehungen stellen gerichtete
  Kanten dar. Die Anzahl der generierten Entit�ten wird �ber Verh�ltnisse vom Tester konfiguriert. Auch dieser Algorithmus
  ist eher f�r die Erzeugung von Massen-Daten geeignet.


  \subsection{Fazit}
  Die von dem zu entwickelnden Generator erzeugten Testdaten sollen allerdings �berschaubar und wartbar sein. Dies steht
  in Widerspruch mit einer Massen-Daten-Generierung, wie sie die kommerziellen Werkzeuge bieten. Zum selben Schluss kommt
  Raza in \cite[126]{CREATINGDATASETS}. Massen-Daten eignen sich eher f�r Stabilit�ts-, Performance- und Regressionstests.
    
  Keines der kommerziellen Werkzeuge ist in der Lage, die Anzahl der zu generierenden Entit�ten selbst zu bestimmen. Auch
  der Algorithmus aus \cite{Houkjaer:2006:SRD:1182635.1164254} ist dazu nicht in der Lage.
  
  Aus diesem Grund soll ein Algorithmus entwickelt werden, der Beziehungen nicht nur zuf�llig generiert, sondern m�glichst
  alle Grenzf�lle erzeugt.  �quivalenzklassenbildung und Grenzwertanalyse sind ein bew�hrtes Vorgehen, um die Menge von
  Test-Daten zu reduzieren. 
  


\section{Generierung von Beziehungen}

Der zu entwickelnde Algorithmus �bernimmt das Konzept der �quivalenzklassenbildung und Grenzwertanalyse, um m�glichst
alle notwendigen Beziehungskombinationen zwischen zwei Entit�tstypen zu modellieren. Die Menge der zu generierenden
Entit�ten soll dabei m�glichst gering gehalten werden.

Unterschiedliche Beziehungstypen stellen unterschiedliche Anforderungen an den Daten-Generator. Bin�re Beziehungstypen
lassen sich in die drei Hauptkategorien 1:1, 1:n und n:m einordnen. Die folgenden Abbildungen stellen die zu generierenden
Entit�ten der beiden Entit�tstypen A und B dar. Eine Entit�t wird von einem kleinen Kreis repr�sentiert, ihr 
Entit�tstyp �ber die Spalte festgelegt. Eine Beziehung zwischen zwei Entit�ten wird �ber eine Verbindungsgerade
beschrieben. Grunds�tzlich k�nnen die beiden Typen A und B auch den selben Typen darstellen.

Ganz allgemein lassen sich alle bin�ren Beziehungen als n..N:m..M-Beziehung ansehen. n und m stellen jeweils
untere Grenzen darf, N und M die oberen. Die grundlegende Generierungsstrategie sieht die Generierung der folgenden
vier Kombinationen vor:
\begin{itemize}
  \item n:m
  \item n:M
  \item N:m
  \item N:M
\end{itemize}
Verschiedene F�lle k�nnen redundant sein, falls untere und obere Grenze identisch sind. Auf die Generierung dieser
redundanten Beziehungen kann verzichtet werden.

Die Abbildungen stellen dar, welche Entit�ten und Beziehungen \textit{mindestens} generiert werden sollten, um
die von den Grenzen der Multiplizit�ten bestimmten �quivalenzklassen abzudecken.

  \subsection{Kategorie der 1:1-Beziehungen}
  
  Unter die Kategorie 1:1-Beziehung fallen alle Beziehungstypen, bei denen eine oder keine Entit�t mit genau einer oder
  keiner Entit�t in Beziehung stehen kann.
  
    \subsubsection{1..1:1..1}
    \label{sec:generieren:categories:11to11}
    
    Eine Entit�t des Typs A steht mit genau einer Entit�t des Typs B in Beziehung. Die Anzahl der generierten Entit�ten
    muss �bereinstimmen, es muss mindestens eine Entit�t pro Typ erzeugt werden (siehe Abbildung \ref{img:generierung:11to11}).
    
    \begin{figure}[htbp]
      \centering
       \includegraphics[width=0.55\textwidth]{images/generierung/1-1-to-1-1.png}
      \caption{Beziehungen nach dem Schema 1..1:1..1}\label{img:generierung:11to11}
    \end{figure}

    
    \subsubsection{0..1:1..1}
    \label{sec:generieren:categories:01to11}
    
    Im Gegensatz zu demr vorherigen Beziehungstyp muss bei dieser eine Entit�t nicht zwingend in Beziehung mit einer anderen stehen.
    Abbildung \ref{img:generierung:01to11} zeigt die zu generierenden Entit�ten der Typen A und B, wobei jede Entit�t von A
    mit einer Entit�t von B in Beziehung stehen muss, eine Entit�t von B jedoch nicht zwingend mit einer Entit�t von A.
    Daraus folgt, dass es von B mindestens eine Entit�t mehr geben muss als von A.

    \begin{figure}[htbp]
      \centering
       \includegraphics[width=0.55\textwidth]{images/generierung/0-1-to-1-1.png}
      \caption{Beziehungen nach dem Schema 0..1:1..1}\label{img:generierung:01to11}
    \end{figure}

    Der Generator muss mindestens zwei Entit�ten des Typs B erzeugen, und eine des Typs A, um sicherzustellen, dass alle
    F�lle f�r diese Beziehung abgedeckt sind. 
    
    Die Beziehung 1..1:0..1 ist symmetrisch zu dieser.
    

    \subsubsection{0..1:0..1}
    \label{sec:generieren:categories:01to01}
    
    Wenn f�r beide Entit�tstypen die Beziehung optional ist, muss der Generator jeweils mindestens 2 Entit�ten erzeugen. Jeweils
    eine Entit�t ohne Beziehung und jeweils eine Entit�t mit einer Beziehung (siehe Abbildung \ref{img:generierung:01to01}).

    \begin{figure}[htbp]
      \centering
       \includegraphics[width=0.55\textwidth]{images/generierung/0-1-to-0-1.png}
      \caption{Beziehungen nach dem Schema 0..1:0..1}\label{img:generierung:01to01}
    \end{figure}

  \subsection{Kategorie der 1:n-Beziehungen}
  
  Eine Entit�t steht in Beziehung mit keiner oder mehreren anderen Entit�ten. Dabei kann die Anzahl begrenzt sein (konkreter
  Wert f�r n) oder unbegrenzt. Der Generator kann nur eine begrenzte Anzahl von Entit�ten erzeugen, die Grenze sollte
  konfigurierbar sein.
  
    \subsubsection{1..1:1..n}
    \label{sec:generieren:categories:11to1n}
    
    Die einfachste Form der 1:n-Beziehungen. Eine Entit�t des Typs A ist in einer Beziehung mit einer oder mehreren Entit�ten
    des Typs B. Eine Entit�t des Typs B ist mit genau einer Entit�t des Typs A in Beziehung. Die zu generierenden Entit�ten 
    sind in Abbildung \ref{img:generierung:11to1n} dargestellt.

    \begin{figure}[htbp]
      \centering
       \includegraphics[width=0.55\textwidth]{images/generierung/1-1-to-1-n.png}
      \caption{Beziehungen nach dem Schema 1..1:1..n}\label{img:generierung:11to1n}
    \end{figure}

    \subsubsection{0..1:1..n}
    \label{sec:generieren:categories:01to1n}
    
    Eine Entit�t des Typs A steht in Beziehung mit einer oder mehreren Entit�ten des Typs B. Eine Entit�t des Typs B steht
    entweder mit genau einer oder mit keiner Entit�t des Typs A in Beziehung. Abbildung \ref{img:generierung:01to1n} stellt
    die zu generierenden Entit�ten dar. 

    \begin{figure}[htbp]
      \centering
       \includegraphics[width=0.55\textwidth]{images/generierung/0-1-to-1-n.png}
      \caption{Beziehungen nach dem Schema 0..1:1..n}\label{img:generierung:01to1n}
    \end{figure}
    
    \subsubsection{1..1:0..n}
    \label{sec:generieren:categories:11to0n}
    
    Eine Entit�t des Typs A steht in Beziehung mit keiner, einer oder mehreren Entit�ten des Typs B. Eine Entit�t des Typs B
    muss mit genau einer Entit�t des Typs A in Beziehung stehen. 

    \begin{figure}[htbp]
      \centering
       \includegraphics[width=0.55\textwidth]{images/generierung/1-1-to-0-n.png}
      \caption{Beziehungen nach dem Schema 1..1:0..n}\label{img:generierung:11to0n}
    \end{figure}



    \subsubsection{0..1:0..n}
    \label{sec:generieren:categories:01to0n}
    
    Eine Entit�t des Typs A steht mit keiner, einer oder mehreren Entit�ten des Typs B in Beziehung. Eine Entit�t des Typs B
    kann mit keiner oder genau einer Entit�t des Typs A in Beziehung stehen.

    \begin{figure}[htbp]
      \centering
       \includegraphics[width=0.55\textwidth]{images/generierung/0-1-to-0-n.png}
      \caption{Beziehungen nach dem Schema 0..1:0..n}\label{img:generierung:01to0n}
    \end{figure}

  \subsection{Kategorie der n:m-Beziehungen}
  \label{sec:generieren:beziehungen:nm}
    
  Die allgemeinste Form einer Beziehung zwischen zwei Eintit�tstypen stellt eine n..N:m..M-Beziehung dar. Dabei handelt es
  sich bei n und m jeweils um untere und bei N und M jeweils um obere Schranken. Untere und obere Schranken k�nnen
  identisch sein.
  
  Jede der in diesem Abschnitt beschriebenen Beziehungsformen stellen Spezialf�lle von n:m-Beziehungen dar, bei denen
  eine oder beide oberen Grenzen 1 sind. Sind beide obere Grenzen gr��er als 1, dann wird eine solche Beziehung
  �blicherweise �ber assoziative Tabellen realisiert.
  
  Sollte eine der unteren Grenzen 0 sein, wird sie als 0..1 interpretiert. D.h es wird eine entsprechende Entit�t ohne
  Beziehung erzeugt, aber auch eine einzelne Entit�t, die in Beziehung mit anderen Entit�ten steht 
  (siehe dazu Abbildungen \ref{img:generierung:1to01} und \ref{img:generierung:3to01}).
  
  Um die unterschiedlichen Kombinationen aus n, N, m und M in den Grafiken zu verdeutlichen, wird die Generierung
  einer 1..3:0..4-Beziehung dargestellt.
  
    \subsubsection{1. Fall (n:m): 1:0..1}
    
    
    Im ersten Fall werden beide unteren Grenzen betrachtet. Da eine der Grenzen 0 ist, wird eine Entit�t ohne Beziehung
    generiert. Abbildung \ref{img:generierung:1to01} visualisiert die zu erzeugenden Entit�ten f�r die Kombination n:m.
    
    \begin{figure}[htbp]
      \centering
       \includegraphics[width=0.55\textwidth]{images/generierung/nm-1-to-0-1.png}
      \caption{Beziehungen nach dem Schema 1:0..1 (n:m)}\label{img:generierung:1to01}
    \end{figure}    
    
    
    \subsubsection{2. Fall (n:M): 1:4}
    
    Der zweite Fall betrachtet eine untere mit einer oberen Grenze. Die in diesem Beispiel erzeugten Entit�ten zeigt
    Abbildung \ref{img:generierung:1to4}.
    
    \begin{figure}[htbp]
      \centering
       \includegraphics[width=0.55\textwidth]{images/generierung/nm-1-to-4.png}
      \caption{Beziehungen nach dem Schema 1:4 (n:M)}\label{img:generierung:1to4}
    \end{figure}    

    \subsubsection{3. Fall (N:m): 3:0..1}
    
    Wie im ersten Fall ist hier eine der beiden Grenzen 0. Die in der Abbildung \ref{img:generierung:3to01} dargestellte 
    Entit�t ohne Beziehung ist der Vollst�ndigkeit halber aufgezeigt, muss aber nicht generiert werden, da sie bereits 
    generiert wurde (siehe Abbildung \ref{img:generierung:1to01}).
    
    \begin{figure}[htbp]
      \centering
       \includegraphics[width=0.55\textwidth]{images/generierung/nm-3-to-0-1.png}
      \caption{Beziehungen nach dem Schema 3:0..1 (N:m)}\label{img:generierung:3to01}
    \end{figure}    

    \subsubsection{4. Fall (N:M): 3:4}
    
    Der vierte Fall behandelt schlie�lich die beiden oberen Grenzen. Abbildung \ref{img:generierung:3to4} zeigt die
    Vollvermaschung der zu erzeugenden Entit�ten.
  
    \begin{figure}[htbp]
      \centering
       \includegraphics[width=0.55\textwidth]{images/generierung/nm-3-to-4.png}
      \caption{Beziehungen nach dem Schema 3:4 (N:M)}\label{img:generierung:3to4}
    \end{figure}

    \subsubsection{Gesamte Generierung (n..N:m..M): 1..3:0..4}
    
    F�r eine Beziehung nach dem Schema 1..3:0..4 w�rde der Algorithmus die in Abbildung \ref{img:generierung:13to04}
    dargestellte Entit�ten und Beziehungen vorsehen.
    
    \begin{figure}[htbp]
      \centering
       \includegraphics[width=0.55\textwidth]{images/generierung/nm-1-3-to-0-4.png}
      \caption{Beziehungen nach dem Schema 1..3:0..4 (n..N:m..M)}\label{img:generierung:13to04}
    \end{figure}

\section{Komplexit�t bei der Generierung von Beziehungen}
\label{sec:generieren:komplexitaet}

Im vorausgegangenen Abschnitt wurden nur Beziehungen zwischen zwei Entit�tstypen betrachtet. In realen Anwendungen k�nnen
Entit�tstypen mit mehr als nur einem anderen Entit�tstyp in Beziehung stehen und auch mit dem selben Entit�tstyp mehr als
nur ein Mal.

Bez�glich der Datengenerierung lassen sich hierbei zwei generelle Vorgehensweisen lassen sich hier unterscheiden:
\begin{itemize}
  \item \textbf{Beziehungen unabh�ngig betrachten}: Es wird angenommen, dass unterschiedliche Beziehungen voneinander
    unabh�ngig sind. Die Frage, ob ein Professor eine Lehrveranstaltung leitet, l�sst keine R�ckschl�sse zu, ob und welche
    Pr�fungen er beaufsichtigt.
  
  \item \textbf{Beziehungen abh�ngig voneinander betrachten}: In der Praxis beeinflussen sich viele Beziehungen. Ein Student,
    der die Pr�fung einer Lehrveranstaltung schreibt, darf wohl nicht gleichzeitig Tutor dieser Veranstaltung sein.
  
\end{itemize}

Alle Beziehungen abh�ngig voneinander zu betrachten kann schnell zu exponentiell zunehmenden Testdaten f�hren. 
Einen riesigen Bestand an Daten um f�r Tests unn�tige Beziehungen und schlie�lich auch Entit�ten zu verringern
scheint aufw�ndiger, als einen kleinen Datenbestand um fehlende Beziehungen punktuell zu erweitern. Aus diesem Grund
ber�cksichtigt der Algorithmus keine Beziehungen in Abh�ngigkeit von anderen -- mit Ausnahme von assoziativen
Tabellen.

\section{Algorithmus zur Generierung von Beziehungen}
\label{sec:generieren:algorithmus}

Der entwickelte Algorithmus �bernimmt aus \cite{Houkjaer:2006:SRD:1182635.1164254} die Idee, das Modell als Graphen
zu betrachten und zu traversieren. Bevor der Pseudocode des Algorithmus beschrieben wird, werden einige verwendete
Begriffe beschrieben und das Klassen-Diagramm des Modells vorgestellt, das der Daten-Generator verwendet.

  \subsection{Begriffserkl�rungen}
  Die Beschreibung des Algorithmus verwendet einige Begriffe und Konventionen, die im Folgenden beschrieben werden.
  
    \subsubsection{Knoten und Kanten}
    
    Die Idee, das Datenbank-Modell als Graphen zu betrachten, stammt aus \cite{Houkjaer:2006:SRD:1182635.1164254}. 
    Tabellen stellen die Knoten des Graphs dar. Die Foreign-Key-Beziehungen zwischen zwei Tabellen stellen die Kanten
    des Graphs dar. Eine Kante ist gerichtet, von der Tabelle mit der Foreign-Key-Spalte zur referenzierten Tabelle hin.
    Aufgrund der Richtung hat jede Kante eine Start- und eine Zieltabelle.
    
    Auch wenn es sich um gerichtete Kanten handelt, ist eine Traversierung in beide Richtungen m�glich.
    
  
    \subsubsection{Assoziative Tabellen}
    
    Assoziative Tabellen verbinden zwei Tabellen. Die beiden verbundenen Tabellen werden im Folgenden als linke und rechte
    Tabelle bezeichnet, analog wird von linker und rechter Kante gesprochen.
    
    Die Reihenfolge, in der die Spalten der Tabelle definiert sind, bestimmt die Einteilung in links und rechts. Der erste
    Fremdschl�ssel referenziert die linke Tabelle. Das Ergebnis der Generierung h�ngt jedoch nicht von dieser Einteilung ab.
  
  \subsection{Klassendiagramm}
  
  F�r die Datengenerierung wird das bisherig verwendete Klassen-Modell um eine Klasse \texttt{Kante} erweitert
  (siehe Abbildung~\ref{img:generierung:generatormodel}). Diese Klasse repr�sentiert eine Kante des Graphen. 
  Zu einer Kante geh�rt eine Start- und eine Zieltabelle. Eine Tabelle selbst kann zu beliebig vielen Kanten geh�ren.
  Alle Tabellen eines Datenbankmodells werden zu einer Tabellenliste zusammengefasst.
  
  \begin{figure}[htbp]
    \centering
     \includegraphics[width=0.75\textwidth]{images/generierung/generator_model.pdf}
    \caption{Diagramm der Generator-Modell-Klassen}\label{img:generierung:generatormodel}
  \end{figure}
  
  

  \subsection{Pseudocode}
  \label{sec:generieren:algorithmus:pseudocode}
  
  Der Algorithmus ist in mehrere Teilfunktionen unterteilt. Einige Funktionen werden rekursiv aufgerufen, um den Graphen
  entlang der Kanten zu traversieren. Der Einstiegspunkt stellt die Funktion \texttt{Generiere\_Test\_Daten} dar, die im
  folgenden Abschnitt beschrieben wird.
  
    \subsubsection{Generiere Test-Daten}
    
    Die Funktion \texttt{Generiere\_Test\_Daten} (siehe Listing \ref{listing:GeneriereTestDaten}) ist der Einstiegspunkt
    f�r den Algorithmus zur Generierung von Test-Daten.
    Im ersten Schritt wird die Reihenfolge der Tabellen festgelegt, die als Startpunkte in Frage kommen. Die Liste
    stellt sicher, dass auch Datenbank-Modelle, die aus mehreren unabh�ngigen Graphen bestehen, vollst�ndig
    generiert werden. In einem Datenbank-Modell, in dem alle Tabellen direkt oder indirekt in Beziehung stehen,
    w�rde die Festlegung einer Starttabelle ausreichen.
    
    Zur Sortierung der Tabellen wird die Anzahl eingehender Kanten verwendet. Der Grund daf�r und die Bedeutung der
    Reihenfolge der Tabellen wird in Abschnitt \ref{sec:generieren:evaluation:tabellenreihenfolge} beschrieben.
    
    Anschlie�end wird �ber diese Tabellenliste iteriert. Wurde eine Tabelle noch nicht behandelt, wird die
    entsprechende Funktion zur Generierung der Daten aufgerufen. Dabei werden nicht-assoziative und
    assoziative Tabellen unterschiedlich behandelt. Der Grund daf�r ist, dass assoziative Tabellen ein
    Hilfskonstrukt darstellen, das selbst eine Beziehung auf ER-Ebene darstellt.
    
    Am Ende stellt der Algorithmus sicher, dass jede Entit�t g�ltig generiert wurde, also dass die Beziehungen
    die Constraints erf�llen. Gegebenenfalls werden hier Entit�ten nach generiert.
  
\begin{lstlisting}[caption=Generiere Test-Daten, label=listing:GeneriereTestDaten]
GeneriereTestDaten
------------------
L := Tabellenliste, nach Anzahl eingehender Kanten aufsteigend geordnet
FOR EACH (noch nicht besuchte Tabelle T IN der geordneten Tabellenliste L)
DO
  Markiere Tabelle T als besucht;
  IF (Tabelle T ist keine assoziative Tabelle)
  THEN CALL Generiere_Daten_Fuer_Nichtassoziative_Tabelle(T);
  ELSE CALL Generiere_Daten_Fuer_Assoziative_Tabelle(T);
  END IF;
END FOR;
CALL Erweitere_Generierte_Daten_Zu_Konsistenten_Daten()
\end{lstlisting}

    \subsubsection{Generiere Daten f�r nichtassoziative Tabelle}
    
    Der Generator betrachtet zur Generierung von Entit�ten die Beziehungen der Tabellen. Aus diesem Grund
    erzeugt die Funktion \texttt{Generiere\_Daten\_Fuer\_Nichtassoziative\_Tabelle}
    (siehe Listing \ref{listing:GeneriereDatenFuerNichtassoziativeTabelle})
    selbst keine Daten. Stattdessen werden die (noch unbehandelten) Kanten der Tabelle betrachtet. 
    
    Handelt es sich bei der an der Beziehung beteiligten Tabelle um eine assoziative Tabelle, wird mit der
    Generierung der assoziativen Beziehungen fortgesetzt. Ansonsten wird die Funktion zur Generierung der
    Daten f�r eine Kante aufgerufen.
    
    Der Algorithmus aus \cite{Houkjaer:2006:SRD:1182635.1164254} bevorzugt ausgehende Kanten bei der Erzeugung
    von Daten, da sich diese mit weniger Aufwand abarbeiten lassen. Diese Bevorzugung wird f�r den hier
    entwickelten Algorithmus �bernommen, auch wenn sich der Aufwand f�r ein- und ausgehende Kanten nicht
    unterscheidet.

\begin{lstlisting}[caption=GeneriereDatenFuerNichtassoziativeTabelle, label=listing:GeneriereDatenFuerNichtassoziativeTabelle]
Generiere_Daten_Fuer_Nichtassoziative_Tabelle(Tabelle T)
---------------------------------------------------
FOR EACH (ausgehende Kante K, die noch nicht besucht wurde)
DO
  IF (Zieltabelle Z von Kante K ist keine assoziative Tabelle)
  THEN Markiere Kante K als besucht;
       CALL Generiere_Daten_Fuer_Kante(K);
       IF (Zieltabelle Z noch nicht besucht) 
       THEN Markiere Tabelle Z als besucht;
            CALL Generiere_Daten_Fuer_Nicht_Assoziative_Tabelle(Z);
       END IF;
  ELSE CALL Generiere_Daten_Fuer_Assoziative_Tabelle(Z);
  END IF;
END FOR;
FOR EACH (eingehende Kante K, die noch nicht besucht wurde)
DO
  IF (Starttabelle S von Kante K ist keine assoziative Tabelle)
  THEN Markiere Kante K als besucht;
       CALL Generiere_Daten_Fuer_Kante(K);
       IF (Starttabelle S noch nicht besucht) 
       THEN Markiere Tabelle S als besucht;
            CALL Generiere_Daten_Fuer_Nicht_Assoziative_Tabelle(S);
       END IF;
  ELSE CALL Generiere_Daten_Fuer_Assoziative_Tabelle(S);
  END IF;
END FOR;
\end{lstlisting}

    \subsubsection{Generiere Daten f�r Kante}
    
    Die Funktion zum Generieren von Daten f�r eine Kante setzt die f�hrt die 
    \ref{sec:generieren:beziehungen:nm} beschriebenen Schritte um (siehe Listing \ref{listing:GeneriereDatenFuerKante}). 
    Es werden alle vier Kombinationen aus Minimum und Maximum behandelt (Zeilen 6-9). Die eigentliche Generierung findet
    ist in eine Hilfsfunktion ausgelagert, die mit allen Min-Max-Kombinationen aufgerufen wird.

\begin{lstlisting}[caption=Generiere Daten Fuer Kante, label=listing:GeneriereDatenFuerKante]
Generiere_Daten_Fuer_Kante(Kante K)   
------------------------------
S := Starttabelle von Kante K;
Z := Zieltabelle von Kante K;
// Generierung der Daten entsprechend Abschnitt 6.2.3
CALL Generiere_Entitaeten_Und_Beziehungen(K, min(S), min(Z));
CALL Generiere_Entitaeten_Und_Beziehungen(K, min(S), max(Z));
CALL Generiere_Entitaeten_Und_Beziehungen(K, max(S), min(Z));
CALL Generiere_Entitaeten_Und_Beziehungen(K, max(S), max(Z));
\end{lstlisting}

    \subsubsection{Generiere Entit�ten und Beziehungen}
    
    Es soll vermieden werden, unn�tige Entit�ten und Beziehungen zu generieren. Unn�tig sind vor allem
    redundante Beziehungen. Diese k�nnen beispielsweise entstehen, wenn die untere und die obere Grenze einer
    Multiplizit�t identisch sind. Deshalb wird f�r jede Kombination aus Kante, Anzahl der beteiligten Entit�ten
    der Starttabelle und Anzahl der beteiligten Entit�ten der Zieltabelle die Generierung nur ein einziges
    Mal durchgef�hrt.
    
    Der Algorithmus (siehe Listing \ref{listing:GeneriereEntitaetenUndBeziehungen}) pr�ft, ob es sich um eine
    optionale Beziehung handelt (Zeile 7 und Zeile 10). Eine der Grenzen s bzw. z
    ist in einem solchen Fall gleich 0. Handelt es sich um eine optionale Beziehung, wird eine entsprechende
    Entit�t berechnet, die in Bezug auf die Kante in keiner Beziehung ist. Anschlie�end wird die Funktion
    rekursiv aufgerufen, diesmal mit dem Wert 1 als Grenze anstelle der 0 (Zeile 9 und Zeile 12).
    
    Sofern s und z beide ungleich 0 sind, wird eine Entit�t in der Zieltabelle berechnet und s Entit�ten 
    in der Starttabelle. Zwischen diesen Entit�ten wird die von der Kante K repr�sentierte Beziehung
    hergestellt (Zeilen 14 bis 19).
    
    Der Eingangswert z kann aufgrund der Tatsache, dass es sich um eine nicht-assoziative Tabelle handelt,
    nur 0 oder 1 sein.
    
\begin{lstlisting}[caption=Generiere Entit�ten Und Beziehungen, label=listing:GeneriereEntitaetenUndBeziehungen]
Generiere_Entitaeten_Und_Beziehungen(K, s, z)
-----------------------------------------
// Sicherstellen, dass nicht mehr Entit�ten als notwendig erzeugt werden
IF (F�r die Kombination (K, s, z) wurden bereits Entit�ten erzeugt) 
THEN RETURN;
END IF
IF (Wert der Grenze s ist 0)
THEN Berechne Entit�t e in Zieltabelle, die in keiner Beziehung zur Starttabelle stehen darf;
     CALL Generiere_Entitaeten_Und_Beziehungen(k, 1, z);
ELSE IF (Wert der Grenze z ist 0)
THEN Berechne Entit�t e in Starttabelle, die in keiner Beziehung zur Zieltabelle stehen darf;
     CALL Generiere_Entitaeten_Und_Beziehungen(k, s, 1);
ELSE // z ist hier immer 1
     Berechne Entit�t e in Zieltabelle, die in keiner Beziehung zur Starttabelle stehen darf;
     FOR i = 1 TO s
     DO 
       Berechne Entit�t se in Starttabelle, die in keiner Beziehung zur Zieltabelle stehen darf;
       Stelle Beziehung zwischen Entit�t se und Entit�t e her;
     END FOR;
END IF;
\end{lstlisting}

    \subsubsection{Generiere Daten f�r assoziative Tabelle}
    
    Assoziative Tabellen werden typischerweise zur Modellierung von n..N:m..M-Beziehungen verwendet. Diese Beziehungen
    werden entsprechend der in Abschnitt \ref{sec:generieren:beziehungen:nm} beschriebenen Strategie generiert. D.h. es
    werden alle vier Kombinationen aus den jeweiligen Minimal- und Maximalwerten betrachtet.
    
    Nach der Erzeugung der Daten f�r die assoziative Tabelle versucht der Algorithmus (siehe 
    Listing \ref{listing:GeneriereDatenFuerAssoziativeTabelle}), die Generierung bei den beiden
    assoziierten Tabellen fortzusetzen -- falls diese noch nicht behandelt wurden.

\begin{lstlisting}[caption=Generiere Daten f�r assoziative Tabelle, label=listing:GeneriereDatenFuerAssoziativeTabelle]
Generiere_Daten_Fuer_Assoziative_Tabelle(Tabelle T)
-----------------------------------------------
LK := linke Kante der assoziativen Tabelle T;
RK := rechte Kante der assoziativen Tabelle T;
markiere Kanten LK und RK als besucht;
LM := Multiplizit�t der linken Beziehung, ausgehende Seite
RM := Multiplizit�t der rechten Beziehung, ausgehende Seite
CALL Generiere_Assoziative_Entitaeten_Und_Beziehungen(T, min(LM), min(RM));
CALL Generiere_Assoziative_Entitaeten_Und_Beziehungen(T, min(LM), max(RM));
CALL Generiere_Assoziative_Entitaeten_Und_Beziehungen(T, max(LM), min(RM));
CALL Generiere_Assoziative_Entitaeten_Und_Beziehungen(T, max(LM), max(RM));
LT := Zieltabelle von Kante LK;
RT := Zieltabelle von Kante RK;
IF (LT wurde noch nicht besucht)
THEN Markiere Tabelle LT als besucht;
     IF (Tabelle LT ist keine assoziative Tabelle)
     THEN CALL Generiere_Daten_Fuer_Nichtassoziative_Tabelle(LT);
     ELSE CALL Generiere_Daten_Fuer_Assoziative_Tabelle(LT);
     END IF;
END IF;
IF (RT wurde noch nicht besucht)
THEN Markiere Tabelle RT als besucht;
     IF (Tabelle RT ist keine assoziative Tabelle)
     THEN CALL Generiere_Daten_Fuer_Nichtassoziative_Tabelle(RT);
     ELSE CALL Generiere_Daten_Fuer_Assoziative_Tabelle(RT);
     END IF;
END IF;
\end{lstlisting}

    \subsubsection{Generiere assoziative Entit�ten und Beziehungen}
    
    Die Funktion \texttt{Generiere\_Assoziative\_Entitaeten\_Und\_Beziehungen} erzeugt Entit�ten und Beziehungen
    in Verbindung mit assoziativen Tabellen (siehe Listing \ref{listing:GeneriereAssoziativeEntitaetenUndBeziehungen}).
    D.h. sie erzeugt Entit�ten in der assoziativen Tabelle und bei Bedarf in den beiden an der Assoziation beteiligten
    Tabellen (linke und rechte Tabelle der assoziativen Tabelle).
    
    Die Funktion erwartet drei Parameter:
    \begin{itemize}
      \item \textbf{Tabelle \texttt{BEAUFSICHTIGT}}: Die assoziative Tabelle, die behandelt wird.

      \item \textbf{Grenze \texttt{l}}: Der momentane Grenzwert der linksseitigen Multiplizit�t.

      \item \textbf{Grenze \texttt{r}}: Der momentane Grenzwert der rechtsseitigen Multiplizit�t.
    \end{itemize}
    
    Abbildung \ref{img:generierung:tauschat} veranschaulicht die Parameter und zeigt, dass die angegebenen Grenzen
    bestimmen, wie viele Entit�ten der \textit{anderen} Tabelle ben�tigt werden. D.h. \texttt{l} bestimmt, wie viele
    Entit�ten der rechten Tabelle, und \texttt{r} bestimmt, wie viele Entit�ten der linken Tabelle generiert werden
    m�ssen.
    
    \begin{figure}[htbp]
      \centering
       \includegraphics[width=0.55\textwidth]{images/generierung/tausch-at.png}
      \caption{Parameter f�r Daten-Generierung bei assoziativer Tabelle}\label{img:generierung:tauschat}
    \end{figure}
    
    F�r den Fall dass \texttt{l} oder \texttt{r} den Wert 0 enthalten, wird eine entsprechende Entit�t in der
    rechten bzw. linken Tabelle berechnet, die keine Beziehung zur assoziativen Tabelle \texttt{AT} hat. 
    In den Zeilen 6 und 10 wird die Anzahl der zu berechnenden Entit�ten der linken Tabelle (Variable \texttt{LA})
    und der rechten Tabelle (Variable \texttt{RA}) berechnet. Als Minimalwert wird wie schon vorher bei
    optionalen Beziehungen der Wert 1 verwendet.
    
    Sollten f�r die Kombination \texttt{LA} und \texttt{RA} f�r die assoziative Tabelle \texttt{AT} bereits
    Daten generiert worden sein, bricht die Funktion an dieser Stelle ab (Zeilen 14 und 15). Ein solcher Fall
    kann eintreten, wenn untere und obere Grenze einer Multiplizit�t identisch sind, oder es eine Multiplizit�t
    der Form 0..1 ist.
    
    Der Algorithmus berechnet zuerst alle Entit�ten in der linken (Zeilen 17 bis 20) und dann in der
    rechten Tabelle (Zeilen 21 bis 24). Danach werden die Entit�ten in der assoziativen Tabelle erzeugt
    und die Beziehungen hergestellt (Zeilen 25 bis 33).
    

\begin{lstlisting}[caption=Generiere Assoziative Entitaeten Und Beziehungen, label=listing:GeneriereAssoziativeEntitaetenUndBeziehungen]
Generiere_Assoziative_Entitaeten_Und_Beziehungen(AT, l, r)
----------------------------------------------------
LK := linke Kante der assoziativen Tabelle AT;
RK := rechte Kante der assoziativen Tabelle AT;
// Grenzen "tauschen" und Mindestanzahl auf 1
LA := Max(r, 1);  // r als Grenze f�r linke Tabelle, mindestens 1
IF (r ist gleich 0) 
THEN Berechne Entit�t in linker Tabelle, die in keiner Beziehung zur assoziativen Tabelle AT stehen darf;
ENDIF;
RA := Max(l, 1);  // l als Grenze f�r rechte Tabelle, mindestens 1
IF (l ist gleich 0) 
THEN Berechne Entit�t in rechter Tabelle, die in keiner Beziehung zur assoziativen Tabelle AT stehen darf;
ENDIF;
IF (F�r die Kombination (AT, LA, RA) wurden bereits Entit�ten erzeugt) 
THEN RETURN;
END IF
FOR i = 1 TO RA
DO 
  Berechne Entit�t le[i] in linker Tabelle, die in keiner Beziehung zur assoziativen Tabelle AT stehen darf;
END FOR;
FOR j = 1 TO LA
DO 
  Berechne Entit�t re[j] in rechter Tabelle, die in keiner Beziehung zur assoziativen Tabelle AT stehen darf;
END FOR;
FOR i = 1 TO RA
DO 
  FOR j = 1 TO LA
  DO 
    Erzeuge Entit�t e in assoziativer Tabelle AT;
    Stelle Beziehung zwischen Entit�t e und Entit�t le[i] her;
    Stelle Beziehung zwischen Entit�t e und Entit�t re[j] her;
  END FOR;
END FOR;
\end{lstlisting}

    \subsubsection{Erweitere Generierte Daten zu konsistenten Daten}
    
    Es kann passieren, dass am Ende der Traversierung des Graphs ung�ltige Entit�ten vorhanden sind.
    Ung�ltig bedeutet, dass sie bei eine oder mehrere Kanten nicht in ausreichend vielen Beziehungen steht.
    Dies kann passieren, wenn f�r eine bereits besuchte Tabelle zus�tzliche Entit�ten erzeugt werden
    m�ssen -- beispielsweise wenn das Datenbank-Modell zirkul�re Abh�ngigkeiten enth�lt.
    
    Der Algorithmus sieht nicht vor, bei zus�tzlicher Erzeugung von Entit�ten den Graphen r�ckw�rts zu
    traversieren. Stattdessen werden am Ende alle Entit�ten validiert und gegebenenfalls wird nachgeneriert.
    
    Die Funktion zur Validierung iteriert �ber die Entit�ten aller Tabellen. Es wird �berpr�ft, ob die
    jeweilige Entit�t g�ltig generiert wurde auf Hinblick der aus- und eingehenden Kanten. Sollte dies f�r
    eine Entit�t nicht der Fall sein, wird eine entsprechende Beziehung hergestellt und die Validierung
    erneut von Anfang an durchgef�hrt.
    
    Listing \ref{listing:ErweitereGenerierteDatenZuKonsistentenDaten} zeigt den Pseudo-Code dieser Funktion.
    
    Es kann passieren, dass diese Funktion niemals beendet wird. Dieses Problem wird in
    Abschnitt~\ref{sec:generieren:offenepunkte:unerfuellbar} thematisiert.
    
\begin{lstlisting}[caption=ErweitereGenerierteDatenZuKonsistentenDaten, label=listing:ErweitereGenerierteDatenZuKonsistentenDaten]
Erweitere_Generierte_Daten_Zu_Konsistenten_Daten()
-------------------------------------------------
FOR EACH (Tabelle T in Tabellenliste)
DO
  FOR EACH (Entit�t e aus generierten Entit�ten der Tabelle T)
  DO
    FOR EACH (Kante K aus ausgehenden Kanten der Tabelle T)
    DO
      IF (e erf�llt Constraints f�r Kante K nicht)
      THEN Berechne Entit�t f in Zieltabelle der Kante K, die in keiner Beziehung zur Tabelle T stehen darf;
           Stelle Beziehung zwischen Entit�t e und Entit�t f her;
           Funktion Erweitere_Generierte_Daten_Zu_Konsistenten_Daten neu starten;
      END IF;
    END FOR;
    FOR EACH (Kante K aus eingehenden Kanten der Tabelle T)
    DO
      IF (e erf�llt Constraints f�r Kante K nicht)
      THEN Berechne Entit�t f in Starttabelle der Kante K, die in keiner Beziehung zur Tabelle T stehen darf;
           Stelle Beziehung zwischen Entit�t e und Entit�t f her;
           Funktion Erweitere_Generierte_Daten_Zu_Konsistenten_Daten neu starten;
      END IF;
    END FOR;
  END FOR;
END FOR;
\end{lstlisting}




  \subsection{Beispiel}
  
  Der Algorithmus soll an einem kleinen Beispiel veranschaulicht werden. Dieses stellt einen Teil des Modells
  des fortlaufenden Beispiels dar. Das reduzierte Datenbank-Modell besteht aus den vier Tabellen RAUM, PROFESSOR,
  BEAUFSICHTIGT und PRUEFUNG (siehe Abbildung \ref{img:generierung:examplemodel}). Die Tabelle BEAUFSICHTIGT ist
  assoziativ. Die Beziehungen zwischen den Tabellen haben im Vergleich zum fortlaufenden Beispiel ver�nderte
  Multiplizit�ten. 
  
  Das Modell l�sst sich auf ER-Ebene folgenderma�en beschreiben:
  Ein Raum kann beliebig vielen Professoren zugeordnet sein. Umgekehrt kann ein Professor genau einem oder keinem
  Raum zugeordnet sein.  Eine Professor kann beliebig viele Pr�fungen beaufsichtigen. Eine Pr�fung ben�tigt
  mindestens drei und h�chstens f�nf Professoren zur Aufsicht.
  

  \begin{figure}[htbp]
    \centering
     \includegraphics[width=0.55\textwidth]{images/generierung/example_model.pdf}
    \caption{Einfaches Beispiel-Modell f�r den Algorithmus}\label{img:generierung:examplemodel}
  \end{figure}
  
  Als konkreter Wert f�r die offene Grenze \texttt{*} wird der Wert 5 verwendet.
  
  Die generierten Daten f�r das vollst�ndige fortlaufende Beispiel befinden sich in Anhang~\ref{chap:anhang:generiertedsl}.
  

    \subsubsection{1. Schritt: Sortieren der Tabellen}
    
    Im ersten Schritt werden die Tabellen nach den eingehenden Kanten aufsteigend sortiert. Abbildung
    \ref{img:generierung:examplemodeledges} zeigt die Kanten des Graphen. Die Zahl der eingehenden Kanten ist
    unter der jeweiligen Tabelle vermerkt.

    \begin{figure}[htbp]
      \centering
       \includegraphics[width=0.55\textwidth]{images/generierung/example_model_edges.pdf}
      \caption{Kanten des einfachen Beispiel-Modells}\label{img:generierung:examplemodeledges}
    \end{figure}
      
    Da zwei Knoten die selbe Anzahl eingehender Kanten haben, wird als Sekund�rkriterium die Reihenfolge
    der Tabellendefinition verwendet. RAUM wurde vor der assoziativen Tabelle BEAUFSICHTIGT definiert und ist
    deshalb in der Ordnung weiter oben.
    
    Die sortierte Tabellenliste stellt sich wie folgt dar:
    \begin{enumerate}
      \item \textbf{RAUM} (0 eingehende Kanten)
      \item \textbf{BEAUFSICHTIGT} (0)
      \item \textbf{PRUEFUNG} (1)
      \item \textbf{PROFESSOR} (2)
    \end{enumerate}
    
    Der Algorithmus iteriert �ber diese Liste und erzeugt ausgehend von der jeweiligen Tabelle
    Daten, sofern die Tabelle nicht bereits behandelt wurde. Die erste Tabelle ist RAUM.
    
    \subsubsection{2. Schritt: Generierung f�r RAUM}

    Der Algorithmus befindet sich bei der Generierung bei RAUM (siehe Abbildung \ref{img:generierung:example:step_raum}).
    
    \begin{figure}[htbp]
      \centering
       \includegraphics[width=0.55\textwidth]{images/generierung/example_step_raum.pdf}
      \caption{Schritt: RAUM}\label{img:generierung:example:step_raum}
    \end{figure}

    Es wird �ber die Kanten von RAUM iteriert und sofern die Kante noch nicht behandelt wurde,
    werden Daten f�r die jeweilige Kante erzeugt. RAUM hat nur eine Kante, diese f�hrt zu PROFESSOR und
    wurde noch nicht besucht.
    
    \subsubsection{3. Schritt: Generierung f�r Kante zwischen RAUM und PROFESSOR}
    
    Die Kante repr�sentiert eine 0..1:0..*-Beziehung, die Generierung f�r eine solche Kante ist in Abschnitt 
    \ref{sec:generieren:categories:01to0n} beschrieben. Jeweils eine Entit�t aus RAUM und PROFESSOR steht
		in keiner Beziehung, zwei Entit�ten befinden sich in einer 1:1-Beziehung und eine Entit�t aus RAUM steht mit
		f�nf Entit�ten aus PROFESSOR in Beziehung. In der Summe werden damit 3~Entit�ten f�r RAUM und 7~Entit�ten
    f�r PROFESSOR erzeugt (siehe Tabelle \ref{tab:generierung:raum_professor}).
    
    \begin{table}[ht]
      \caption{Zuordnung der Entit�ten von RAUM und PROFESSOR}
      \centering
      \begin{tabular}{l|l}
        RAUM     & PROFESSOR     \\
        \hline
        RAUM\_1  &            \\
                 & PROF\_1  \\
        RAUM\_2  & PROF\_2  \\
        RAUM\_3  & PROF\_3  \\
        RAUM\_3  & PROF\_4  \\
        RAUM\_3  & PROF\_5  \\
        RAUM\_3  & PROF\_6  \\
        RAUM\_3  & PROF\_7  \\
        \hline
        3 Entit�ten  & 7 Entit�ten  \\
      \end{tabular}
      \label{tab:generierung:raum_professor}
    \end{table}
    
    Nachdem die Generierung der Entit�ten und Beziehungen f�r die Kante abgeschlossen ist, setzt der
    Algorithmus die Arbeit bei der Tabelle fort, die mit dieser Kante verbunden ist: PROFESSOR.

    \subsubsection{4. Schritt: Generierung f�r PROFESSOR}

    Die Generierung der Daten f�r die Kante zwischen RAUM und PROFESSOR ist abgeschlossen, RAUM ist
    als bereits besuchte Tabelle markiert. Abbildung~\ref{img:generierung:example:step_professor} stellt
    den Generierungsstand grafisch dar.

    \begin{figure}[htbp]
      \centering
       \includegraphics[width=0.55\textwidth]{images/generierung/example_step_professor.pdf}
      \caption{Schritt: PROFESSOR}\label{img:generierung:example:step_professor}
    \end{figure}
    
    
    Der Algorithmus iteriert �ber die Kanten von PROFESSOR. Dabei w�rden zuerst ausgehende Kanten betrachtet,
		dann die eingehenden. Die Tabelle hat Tabelle hat nur zwei eingehende, von denen eine bereits besucht wurde. Die
    Traversierung des Graphs wird mit der Kante zu Tabelle BEAUFSICHTIGT fortgesetzt. Bei BEAUFSICHTIGT handelt es sich
		um eine assoziative Tabelle, weshalb der n�chste Schritt die Generierung von Daten f�r BEAUFSICHTIGT darstellt.
    
    \subsubsection{5. Schritt: Generierung f�r BEAUFSICHTIGT}
    
    Der Daten-Generator befindet sich bei der assoziativen Tabelle BEAUFSICHTIGT, die Tabellen RAUM und PROFESSOR
    wurden bereits besucht (siehe Abbildung~\ref{img:generierung:example:step_beaufsichtigt}).

    \begin{figure}[htbp]
      \centering
       \includegraphics[width=0.55\textwidth]{images/generierung/example_step_beaufsichtigt.pdf}
      \caption{Schritt: BEAUFSICHTIGT}\label{img:generierung:example:step_beaufsichtigt}
    \end{figure}

    Die assoziative Tabelle BEAUFSICHTIGT dient der Modellierung einer 0..*:3..5-Beziehung zwischen den beiden
    Tabellen PROFESSOR und PRUEFUNG. Das Generierungsschema f�r assoziative Tabellen ist in 
    Abschnitt~\ref{sec:generieren:beziehungen:nm} beschrieben.
    
    \begin{table}[ht!]
      \caption{Zuordnung der Entit�ten von BEAUFSICHTIGT, PROFESSOR und PRUEFUNG}
      \centering
      \begin{tabular}{l|l|l||l|l|l}
        BEAUF. & PROFESSOR & PRUEFUNG  & BEAUF. & PROFESSOR & PRUEFUNG\\
        \hline
  % 0..1:3                        5:5      
               & PROF\_1   &           & B\_24  & PROF\_13  & PRUEF\_8  \\
        B\_1   & PROF\_2   & PRUEF\_1  & B\_25  & PROF\_13  & PRUEF\_9  \\
        B\_2   & PROF\_3   & PRUEF\_1  & B\_26  & PROF\_13  & PRUEF\_10 \\
        B\_3   & PROF\_4   & PRUEF\_1  & B\_27  & PROF\_13  & PRUEF\_11 \\
  % 0..1:5       
        B\_4   & PROF\_5   & PRUEF\_2  & B\_28  & PROF\_13  & PRUEF\_12 \\
        B\_5   & PROF\_6   & PRUEF\_2  & B\_29  & PROF\_14  & PRUEF\_8  \\
        B\_6   & PROF\_7   & PRUEF\_2  & B\_30  & PROF\_14  & PRUEF\_9  \\
        B\_7   & PROF\_8   & PRUEF\_2  & B\_31  & PROF\_14  & PRUEF\_10 \\
        B\_8   & PROF\_9   & PRUEF\_2  & B\_32  & PROF\_14  & PRUEF\_11 \\
  % 5:3
        B\_9   & PROF\_10  & PRUEF\_3  & B\_33  & PROF\_14  & PRUEF\_12 \\
        B\_10  & PROF\_10  & PRUEF\_4  & B\_34  & PROF\_15  & PRUEF\_8  \\
        B\_11  & PROF\_10  & PRUEF\_5  & B\_35  & PROF\_15  & PRUEF\_9  \\
        B\_12  & PROF\_10  & PRUEF\_6  & B\_36  & PROF\_15  & PRUEF\_10 \\
        B\_13  & PROF\_10  & PRUEF\_7  & B\_37  & PROF\_15  & PRUEF\_11 \\
        B\_14  & PROF\_11  & PRUEF\_3  & B\_38  & PROF\_15  & PRUEF\_12 \\
        B\_15  & PROF\_11  & PRUEF\_4  & B\_39  & PROF\_16  & PRUEF\_8  \\
        B\_16  & PROF\_11  & PRUEF\_5  & B\_40  & PROF\_16  & PRUEF\_9  \\
        B\_17  & PROF\_11  & PRUEF\_6  & B\_41  & PROF\_16  & PRUEF\_10 \\
        B\_18  & PROF\_11  & PRUEF\_7  & B\_42  & PROF\_16  & PRUEF\_11 \\
        B\_19  & PROF\_12  & PRUEF\_3  & B\_43  & PROF\_16  & PRUEF\_12 \\
        B\_20  & PROF\_12  & PRUEF\_4  & B\_44  & PROF\_17  & PRUEF\_8  \\
        B\_21  & PROF\_12  & PRUEF\_5  & B\_45  & PROF\_17  & PRUEF\_9  \\
        B\_22  & PROF\_12  & PRUEF\_6  & B\_46  & PROF\_17  & PRUEF\_10 \\
        B\_23  & PROF\_12  & PRUEF\_7  & B\_47  & PROF\_17  & PRUEF\_11 \\
               &           &           & B\_48  & PROF\_17  & PRUEF\_12 \\
        \hline
               &           &           & 48 Ent. & 17 Entit�ten  & 12 Entit�ten  \\
       \end{tabular}
      \label{tab:generierung:professor_pruefung}
    \end{table}
        
    Tabelle~\ref{tab:generierung:professor_pruefung} zeigt die Zuordnung von Entit�ten aus PROFESSOR
    und PRUEFUNG. Jedes Paar f�hrt zu einer Entit�t in der Tabelle BEAUFSICHTIGT.

    Eine Entit�t aus PROFESSOR steht in keiner Beziehung mit Entit�ten aus PRUEFUNG. Jeweils drei 
    und f�nf Entit�ten PROFESSOR stehen mit genau einer Entit�t aus PROFESSOR in Beziehung (1. und 2. Fall).
    F�nf Entit�ten aus PROFESSOR stehen mit drei Entit�ten aus PRUEFUNG (3. Fall) und weitere f�nf
    Entit�ten aus PROFESSOR schlie�lich mit f�nf Entit�ten aus PRUEFUNG (4. Fall) in Beziehung.
    
    Insgesamt werden f�r die generierten Beziehungen 17~Entit�ten aus PROFESSOR, 12~Entit�ten aus
    PRUEFUNG und 48~Entit�ten in der assoziativen Tabelle BEAUFSICHTIGT ben�tigt. 7~Entit�ten aus PROFESSOR
    wurden bereits in Schritt 3 erzeugt, die anderen 10~Entit�ten werden nachgeneriert.
    
    Der Algorithmus hat die Generierung f�r BEAUFSICHTIGT abgeschlossen und setzt die Generierung bei
    PRUEFUNG fort, da PROFESSOR bereits besucht wurde.
    
    \subsubsection{6. Schritt: Generierung f�r PRUEFUNG}
    
    Abbildung \ref{img:generierung:example:step_pruefung} stellt die Ausgangssituation f�r
    den Algorithmus f�r die Generierung von Daten f�r PRUEFUNG dar. Die Tabellen RAUM,
    PROFESSOR und BEAUFSICHTIGT wurden bereits besucht, ebenso die Kanten von PROFESSOR nach RAUM,
    BEAUFSICHTIGT nach PROFESSOR und BEAUFSICHTIGT nach PRUEFUNG.

    \begin{figure}[htbp]
      \centering
       \includegraphics[width=0.55\textwidth]{images/generierung/example_step_pruefung.pdf}
      \caption{Schritt: PRUEFUNG}\label{img:generierung:example:step_pruefung}
    \end{figure}
    
    Es gibt keine unbesuchten Kanten in PRUEFUNG. Der Algorithmus kehrt zu PROFESSOR und
    von dort zu RAUM zur�ck, wo ebenfalls keine unbesuchten Kanten mehr sind. In der Liste
    der Tabellen befindet sich auch keine unbesuchten Tabellen mehr, so dass nun der letzte
    Schritt folgt: Die Erweiterung der Daten sofern notwendig.

    \subsubsection{7. Schritt: Erweitern der Daten sofern notwendig}
    
    In diesem Beispiel wurden einige Entit�ten f�r PROFESSOR in Schritt 5 nachgeneriert. Da
    diese Entit�ten nur eine optionale Beziehung zu einer Entit�t aus RAUM hat, sind die
    nachgenerierten Entit�ten g�ltig. Eine Erweiterung der Daten ist somit nicht notwendig
    und die Generierung abgeschlossen.


\section{Praktischer Einsatz / Evaluation}

  \subsection{Einfluss der Tabellenreihenfolge}
  \label{sec:generieren:evaluation:tabellenreihenfolge}

  Die Art und Weise, wie Beziehungen generiert werden, h�ngen von der Reihenfolge ab, in der die Tabellen und Kanten behandelt
  werden. Die gew�hlte Reihenfolge basiert auf den Kriterien aus \cite{Houkjaer:2006:SRD:1182635.1164254}. Die Sortierung
  wird nicht aus Qualit�tsgr�nden, sondern f�r ein deterministisches Verhalten des Algorithmus beibehalten. Da verschiedene
  Tabellen gleich viele eingehende Kanten haben k�nnen, muss das deterministische Verhalten durch weitere Sortierkritieren
  sichergestellt werden. Dies kann z.B. der Tabellen-Name sein, oder auch die Reihenfolge, in der die Tabellen definiert
  worden sind.
  
  Empirische Versuche f�hrten zu der Vermutung, dass die Reihenfolge keinen Einfluss auf die Anzahl der generierten Entit�ten hat.
  Auf einen Beweis daf�r wird an dieser Stelle verzichtet, da es weniger wichtig ist, die tats�chlich minimale Anzahl an Entit�ten
  zu generieren. Viel wichtiger ist, dass g�ltige DataSets generiert werden.

\section{Konkrete Implementierung und Integration in Toolset}

Die Integration des Daten-Generators f�hrt zu generierungsspezifischen Erweiterungen im Datenbank-Modell. So kann f�r jede Spalte
individuell ein Werte-Generator festgelegt werden -- ansonsten wird ein Standard-Werte-Generator f�r den jeweiligen Datentyp verwendet.
Listing~\ref{listing:modell:ausschnittfuergenerierung} zeigt den Ausschnitt der Modell-Definition des fortlaufenden Beispiels,
erg�nzt um Daten-generierungsspezifische Anweisungen. Ein vollst�ndiges Modell ist in Anhang~\ref{chap:anhang:modell} aufgelistet.
Im Listing wird lediglich f�r eine Spalte ein individueller Werte-Generator festgelegt (Zeile 17).

Die Zufallszahlengeneratoren der Werte-Generatoren werden �ber Seeds initialisiert. Damit der selbe Werte-Generator f�r verschiedene
Spalten standardm��ig unterschiedliche Werte erzeugt, wird der Standard-Spalten-Seed mit Hilfe des Tabellen- und des Spaltennamen
berechnet. Es ist au�erdem m�glich, tabellenspezifische Seeds und ein modellspezifisches Seed festzulegen, die jeweils den Standardwert
0 haben. Das tats�chliche Seed, mit dem die Zufallsgeneratoren initialisiert werden, stellt die Summe aus dem Spalten-Seed,
dem Tabellen-Seed und dem Modell-Seed dar.

Auf diese Weise kann �ber Seeds verh�ltnism��ig einfach gesteuert werden, dass nur eine Spalte mit neuen Zufallswerten generiert wird
(Zeile 19), alle Spalten einer Tabelle (Zeile 11) oder alle Spalten aller Tabellen (Zeile 6). Es ist dar�ber hinaus auch m�glich,
zwei Spalten mit selben Zufallswerten zu erzwingen.

Da der Daten-Generator die Anzahl zu erzeugender Entit�ten �ber die Beziehungen bestimmt, w�rde eine Tabelle ohne eine Beziehung zu anderen
Tabellen leer bleiben. Das Modell wird deshalb f�r jede Tabelle um einen Wert erweitert, der Mindestanzahl der zu generierenden Entit�ten
enth�lt (Zeile 12). Der Standardwert daf�r ist 1.

Au�erdem l�sst sich f�r das Modell festlegen, welcher konkrete Wert anstelle der offenen Grenze \texttt{*} verwendet werden soll (Zeile 7).

\begin{lstlisting}[caption=Ausschnitt des f�r die Daten-Generierung erweiterten Modells, label=listing:modell:ausschnittfuergenerierung]
public HochschuleModel()
{
  database("Hochschule");
  packageName("com.seitenbau.testing.hochschule.model");
  enableTableModelClassesGeneration();
  seed(1);
  infinite(2);

  Table raum = table("raum")
      .description("Die Tabelle mit den R�umen der Hochschule")
      .seed(3)
      .minEntities(20)
      .column("id", DataType.BIGINT)
        .defaultIdentifier()
        .autoInvokeNext()
      .column("gebaeude", DataType.VARCHAR)
        .generator(new GebaeudeGenerator())
      .column("nummer", DataType.VARCHAR)
        .seed(5)
    .build();
  
  ...
}
\end{lstlisting}

\section{Offene Punkte}
\label{sec:generieren:offenepunkte}

Auch wenn der Algorithmus seine Tauglichkeit gezeigt hat, bietet er einige M�glichkeiten zur Verbesserung.
Ein paar Probleme und Herausforderungen werden im Folgenden aufgezeigt.

  \subsection{Abh�ngigkeiten von Beziehungen}
  
  Bereits in Abschnitt \ref{sec:generieren:komplexitaet} wurde das Problem angesprochen, dass Beziehungen
  nicht immer unabh�ngig voneinander sind. Der Algorithmus in dieser Form tr�gt diesem Problem nur bei assoziativen
  Tabellen Sorge.
  

  \subsection{Abh�ngigkeiten von Spaltendaten}
  
  Die Datengeneratoren f�r die Spalten arbeiten unabh�ngig voneinander. Allerdings k�nnen Werte in verschiedenen
  Spalten voneinander abh�ngen:

  \begin{itemize}
    \item \textbf{Vorname} und \textbf{Geschlecht}
    \item \textbf{PLZ} und \textbf{Ort}
    \item \textbf{Start-} und \textbf{Endwerte}, z.B. bei Datumsbereichen, Zeitspannen, Grenzwerten, ...
    \item Werte, die sich aus Werten anderen Spalten \textbf{zusammensetzen}
  \end{itemize}
  
  Au�erdem k�nnen Spaltenwerte sich auch auf Beziehungen zu anderen Entit�ten auswirken:
  
  \begin{itemize}
    \item \textbf{Geschlecht} und \textbf{Eltern-Eigenschaft}: m�nnlich/weiblich muss zur Rolle Vater/Mutter passen
    \item \textbf{Fakult�t} und \textbf{Raum-Nummer}: Das B�ro eines Professors h�ngt von seiner Fakult�t ab.
    \item \textbf{�berschneidungen von Lehrveranstaltungen} eines Professors: Ein Professor kann nicht zwei
      Lehrveranstaltungen lehren, die sich zeitlich �berschneiden.
  \end{itemize}
  

  \subsection{Unerf�llbare Datenbankschemata}
   \label{sec:generieren:offenepunkte:unerfuellbar}
  
  Nicht f�r jedes formal g�ltige Datenbankschema lassen sich alle Beziehungsf�lle generieren. 
  Abbildung~\ref{img:generierung:offen:infinite} zeigt ein zyklisches Datenbankschema mit den drei
  Tabellen A, B und C. Jedes Entit�t aus A steht genau mit einer Entit�t aus B und einer Entit�t
  aus C in Beziehung. Jede Entit�t aus B steht mit genau einer Entit�t aus A und einer Entit�t
  aus C in Beziehung. Jede Entit�t aus C steht mit genau einer Entit�t aus A in Beziehung und
  mit einer oder keiner Entit�t aus B.
  
  \begin{figure}[htbp]
    \centering
     \includegraphics[width=0.55\textwidth]{images/generierung/infinite.png}
    \caption{Formal korrektes aber "`unerf�llbares"' Datenbankschema}\label{img:generierung:offen:infinite}
  \end{figure}

  Aus der Beschreibung geht schon hervor, dass es kein C geben kann, das nicht in Beziehung mit
  einer Entit�t aus B steht. Denn eine Entit�t in C muss mit einer Entit�t in A in Beziehung
  stehen, und eine Entit�t in A schlie�lich mit einer Entit�t in B. Und diese Entit�t in B
  ben�tigt eine Entit�t C f�r eine Beziehung. Es kommt entweder die Entit�t aus C in Frage,
  die nicht mit einer Entit�t aus B in Beziehung steht, oder es muss eine neue Entit�t in C
  generiert werden -- dann beginnt der Zyklus allerdings von vorne.
  
  Der Algorithmus w�rde sich hierbei immer f�r die zus�tzliche Generierung einer weiteren
  Entit�t in C entscheiden und schlie�lich nicht terminieren. Die momentane Implementierung bricht
  die Generierung allerdings nach �berschreiten eines Grenzwerts f�r nachtr�glich generierte
  Entit�ten ab.


	\chapter{Proof of Concept}
\label{chap:proofofconcept}


  \chapter{Zusammenfassung und Ausblick}

\todo{Zusammenfassung und Fazit schreiben}

- Kurze Zusammenfassung: Was wurde erreicht

- Offene Punkte / Weiterführende Arbeiten
  \chapter*{Titel}
\label{chap:titel}

\section*{Untertitel}
\label{Titel:�btertitel}


\subsection*{Stichpunkte}
\begin{itemize}
	\item
		\textbf{Item 1}: 
		Text

	\item
		\textbf{Item 2}: 
		Text
	
\end{itemize}


\subsection*{Aufz�hlung}
\begin{enumerate}
	\item
		\textbf{Item 1}: 
		Text

	\item
		\textbf{Item 2}: 
		Text
	
\end{enumerate}

\subsection*{Abk�rzung}
\nomenclature{GUI}{Graphical User Interface (Grafische Benutzeroberfl�che)}


\subsection*{Quellcode}
\begin{lstlisting}[caption=Der Titel, label=listing:label]
Code
\end{lstlisting}

\subsection*{Verweise}
\begin{enumerate}
	\item siehe \ref{listing:label}
	\item \reflst{listing:label}
	\item \refabb{img:imagelabel}
	\item \refsec{Titel:�btertitel}
	\item \refchap{chap:titel}
\end{enumerate}

\subsection*{Zitate}
\begin{enumerate}
	%\item \cite{MANUAL_ID}
	\item \cite{BASISWISSEN_SOFTWAETEST}
	\item \cite[20ff]{DER_INTEGRATIONSTEST}
	\item \cite{DOMAIN_SPECIFIC_LANGUAGES}
	\item \cite{MODELLGETRIEBENE_SOFTWAREENTWICKLUNG}
	\item \cite{DOMAIN_DRIVEN_DESIGN}
\end{enumerate}

\subsection*{Bild}
\begin{figure}[H]
	\centering
	 \includegraphics[width=0.8\textwidth]{images/cover/htwg_logo.png}
	\caption{Der Titel}\label{img:imagelabel}
\end{figure}


\subsection*{Bildgruppe}
\begin{figure}[htbp]
	\centering
	\subcaptionbox{Text 1\label{img:imagelabelX:image1}}[0.32\textwidth]{
		 \includegraphics[width=0.32\textwidth]{images/cover/htwg_logo.png}
	}
	\subcaptionbox{Text 2\label{img:imagelabelX:image2}}[0.32\textwidth]{
		 \includegraphics[width=0.25\textwidth]{images/cover/htwg_logo.png}
	}
	\subcaptionbox{Text 3\label{img:imagelabelX:image3}}[0.32\textwidth]{
		 \includegraphics[width=0.16\textwidth]{images/cover/htwg_logo.png}
	}
	\caption{Gemeinsamer Titel}\label{img:imagelabelX}
\end{figure}




  
  \phantomsection % n�tig damit das Tabellenverzeichnis korrekt verlinkt wird
  \cleardoublepage
  \printnomenclature
  %\addcontentsline{toc}{chapter}{Abk�rzungsverzeichnis}
  

  %% Abbildungsverzeichnis generieren
  \phantomsection  % n�tig damit das Abbildungsverzeichnis korrekt verlinkt wird
  \cleardoublepage
  \listoffigures
  %\addcontentsline{toc}{chapter}{Abbildungsverzeichnis}
  
  %% Listingverzeichnis generieren
  \phantomsection  % n�tig damit das Listingverzeichnis korrekt verlinkt wird
  \cleardoublepage
  \lstlistoflistings
  %\addcontentsline{toc}{chapter}{Listings}
  
  %% Tabellenverzeichnis generieren
  %\phantomsection % n�tig damit das Tabellenverzeichnis korrekt verlinkt wird
  %\cleardoublepage
  %\listoftables
  %\addcontentsline{toc}{chapter}{Tabellenverzeichnis}

  %% Literaturverzeichnis generieren
  \phantomsection  % n�tig damit das Literaturverzeichnis korrekt verlinkt wird
  \cleardoublepage
  \printbibliography  
  %m\addcontentsline{toc}{chapter}{Literaturverzeichnis}

  %% Abschnitt fuer den Anhang
  \begin{appendix}
  \end{appendix}
	
	\todos
\end{document}