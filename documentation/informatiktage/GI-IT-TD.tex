\documentclass{lni}

\IfFileExists{latin1.sty}{\usepackage{latin1}}{\usepackage{isolatin1}}

\usepackage{graphicx}

\usepackage{graphicx}
%\usepackage{german}
\usepackage{color}
\usepackage[left]{eurosym}
%\usepackage{pdfpages}
%\usepackage[colorlinks=true, citecolor = black, linkcolor = black, urlcolor = blue]{hyperref}
%\usepackage{hyperref}
\usepackage[utf8]{inputenc}
\usepackage{tikz}

\usepackage{listings}                       %% Codelistungs formatieren


%-------------------------------------------------------------------------
%���Listings
%-------------------------------------------------------------------------

% Einstellungen f�r Listings mit Eclipse Code Style.
\usepackage{color}
\usepackage{courier}
\definecolor{sh_comment}{rgb}{0.12, 0.38, 0.18 }     % adjusted, in Eclipse: {0.25, 0.42, 0.30 } = #3F6A4D
\definecolor{sh_keyword}{rgb}{0.37, 0.08, 0.25}      % #5F1441
%\definecolor{sh_string}{rgb}{0.06, 0.10, 0.98}       % #101AF9% Farbe f�r Kommentare
\definecolor{sh_string}{rgb}{1.0, 0.0, 0.8}       % 255, 0, 204
\definecolor{sh_static}{rgb}{0.0, 0.0, 0.75 }      % #0000C0
\def\lstsmallmath{\leavevmode\ifmmode \scriptstyle \else� \fi}
\def\lstsmallmathend{\leavevmode\ifmmode� \else� \fi}

\newcommand{\lstJava}[1]{\lstinline[language=Java,breaklines=true,basicstyle= \listingsfontinline]$#1$}

\newcommand{\lstSetTiny}{\lstset{basicstyle= \tiny\ttfamily}}
\newcommand{\lstSetNormal}{\lstset{basicstyle= \scriptsize\ttfamily}}
\newcommand{\lstSetBig}{\lstset{basicstyle= \ttfamily\mdseries}}

\lstset{
  language=java,
  frame=single, %shadowbox,
  rulesepcolor=\color{black},
  showspaces=false,
	showtabs=false,
	tabsize=2,
  numberstyle=\tiny\ttfamily,
	numbers=left,
  basicstyle=\scriptsize\ttfamily,
  stringstyle=\color{sh_string},
  keywordstyle=\color{sh_keyword}\bfseries,
  commentstyle=\color{sh_comment}\itshape,
  captionpos=b,
  xleftmargin=0.7cm,
	xrightmargin=0.5cm,
  lineskip=-0.1em,
	breaklines=true,
  escapebegin={\lstsmallmath},
	escapeend={\lstsmallmathend},
	morekeywords={def, use}
}

\lstdefinestyle{java}{
    language={java},
    moredelim=**[is][\color{sh_static}]{`}{`},          
    moredelim=**[is][\color{sh_static}\itshape]{@}{@},  
    moredelim=**[is][\color{sh_keyword}\bfseries]{<}{>},
}


%\usepackage{fontspec}                        %% Codelistungs formatieren
%\newfontfamily\listingsfontinline[Scale=0.8]{Courier New} 
%\newfontfamily\listingsfont[Scale=0.7]{Courier}




\author{
	Nikolaus Moll\footnote{Bis 31.12.2012 Student im Master-Studiengang Informatik / akademischer Mitarbeiter der HTWG Konstanz; seit 15.01.2014 tätig bei der PENTASYS AG in München.}
\\ Jürgen Wäsch, Christian Baranowski \\ 
	%Fakult\"at Informatik,
	HTWG Konstanz / SEITENBAU GmbH\\ 
%	Brauneggerstrasse 55 \\ 
%	D-78462 Konstanz \\ 
%	nikol@usmoll.de %
           \texttt{stu@dev.nikolaus-moll.de}\\ \texttt{juergen.waesch@htwg-konstanz.de}\\ \texttt{ christian.baranowski@seitenbau.com} 
%	\{tzink, haase, waesch\}@htwg-konstanz.de
}

\title{Modellierung und Generierung von Test-Daten für Datenbank-basierte Anwendungen\footnote{Diese Arbeit wurde im Seerhein-Lab (\texttt{www.seerhein-lab.org/}) durchgeführt, einer Kooperation der HTWG Konstanz und der Firma SEITENBAU GmbH.}}

%\title{Modellierung und Generierung von Test-Daten f\"ur Datenbank-basierte Anwendungen\footnote{Die Arbeit wurde im Kontext des FHprofUnt-Projektes ''Transparente Integration von NAT-Traversierungstechniken in Java'' durchgef\"uhrt, das vom BMBF und der Seitenbau GmbH in Konstanz gef\"ordert wird. Projektpartner sind die HTWG Konstanz, Seitenbau GmbH und die Universit\"at Konstanz.}}


\begin{document}
\maketitle




\begin{abstract}

%Softwaretests haben sich als Teil der Qualit�tssicherung von Softwareprojekten etabliert.
%F�r Tester ist die Modellierung von Testdaten f�r Datenbank-basierte Anwendungen allerdings
%nicht immer einfach. Die Daten k�nnen aufgrund von Beziehungen von Datens�tzen schnell
%un�bersichtlich und komplex werden. Wegen der Komplexit�t versuchen Tester, mehrere Tests
%mit denselben Daten durchzuf�hren.
%
%In dieser Masterarbeit werden eine neue Modellierungssprache f�r Testdaten f�r
%Datenbank-basierte Anwendungen und ein Algorithmus zur Generierung von Testdaten 
%vorgestellt. Die Sprache erlaubt eine �bersichtliche Beschreibung von Daten und
%von Beziehungen zwischen Datens�tzen. Der Algorithmus zur Generierung erzeugt
%Daten anhand der Beziehungstypen im Datenbank-Modell. Der Algorithmus versucht
%viele Grenzf�lle zu erzeugen, so dass die Daten in m�glichst vielen Tests verwendet
%werden k�nnen.
%\textit{Apache License 2.0}.

\end{abstract}

\clearpage

\section{Problemstellung und Überblick}

%Softwaretests sind ein wichtiger Baustein f�r die Qualit�tssicherung von modernen Softwareprojekten.
%F�r Tests werden Testdaten spezifiziert. Auf deren Basis wird das Verhalten der Software gepr�ft.
%Bei Datenbank-basierten Anwendungen werden Testdaten in der Regel sehr umfangreich und komplex.
%
%Die Komplexit�t ergibt sich aus der Beschreibung von Beziehungen zwischen den einzelnen
%Datens�tzen. Besonders bei Systemen mit komplexen Datenbank-Schemata kann ein
%Testdaten-Modell schnell un�bersichtlich werden.
%
%F�r den Tester sind un�bersichtliche Testdaten aus verschiedenen Gr�nden ein Problem.
%Einerseits machen sie die Pflege fehleranf�llig. Andererseits ist es schwer,
%die modellierten Daten zu erfassen und zu verstehen. Tester w�nschen deshalb h�ufig Daten,
%die f�r mehrere Tests nutzbar sind. Dadurch reicht es aus, nur ein oder zumindest
%wenige Daten-Modelle zu verstehen und zu pflegen.
%
%Die im Rahmen einer Master Thesis [Quelle] entwickelte Modellierungssprache f�r Testdaten erlaubt
%eine �bersichtliche Modellierung von Testdaten. Sie ist einfach zu nutzen und integriert sich
%in g�ngige Entwicklungsumgebungen.
%
%Dar�ber hinaus wurde ein Algorithmus f�r die Generierung von Testdaten konzipiert. Das Ziel des
%Generators ist es, mit wenigen Datens�tzen viele Grenzf�lle bei Beziehungen abzudecken.



%
%\textbf{Aus FORUM Artikel}
%
%Zum Testen von Software-Anwendung, die Daten persistent in einem Datenbanksystem verwalten, werden Testdaten für die Datenbank benötigt. Für Anwendungen mit einem komplexen Datenbankschema ist die Spezifikation dieser Testdaten meist nicht trivial, da neben den Entitäten auch deren Beziehungen betrachtet werden müssen. Diese Beziehungen unterliegen in der Regel einer Menge komplexer fachlicher Regeln, die sich aus dem Domänen-Modell der Anwendung ergeben. 
%
%In dieser Arbeit wird untersucht, wie die zum Test von Datenbank-basierten Anwendungen notwendigen Testdaten auf einfache Weise beschrieben und automatisch erzeugt werden können.
%
%%---
%
%Eine zu prüfende Anwendung wird im Kontext von Software-Tests als System under Test (SUT) bezeichnet. Alle Voraussetzungen für einen Test werden dabei als sog. Test Fixture bezeichnet. Idealerweise soll die in einem Test Fixture beinhaltete Datenmenge für einen funktionalen Test eine hohe Testabdeckung bieten, dabei aber so klein und übersichtlich wie möglich sein. Somit sind die Testdaten einfacher zu verstehen und für eine große Anzahl von funktionalen Tests zu verwenden.
%
%Ein üblicher Ansatz, ein SUT in Verbindung mit einer Datenbank zu testen, ist das Testmuster Back Door Manipulation \cite{XUNIT_TEST_PATTERNS}. Idee hierbei ist, dass das Einspielen des Test Fixture in die Datenbank direkt durch den Test am SUT vorbei, d.h. sozusagen durch eine Hintertür, geschieht. Im ersten Schritt, dem Setup, wird die Datenbank über direkten Zugriff an dem SUT vorbei in den Anfangszustand des Test Fixture gebracht. Anschließend können im Exercise-Schritt die zu testenden Operationen am SUT durchgeführt werden. Die Überprüfung, ob sich das SUT richtig verhalten hat, findet im Verify-Schritt statt. Dabei kann der Zustand der Datenbank mit dem erwarteten Zustand des SUT verglichen, ebenfalls über die Test-eigene Datenbankverbindung. Abschließend können im vierten optionalen Schritt, Teardown, noch Aufräumarbeiten in der Datenbank implementiert sein.
%
%Basis für die nachfolgend beschriebenen Projektarbeiten ist die Java-Bibliothek Simple Test Utils for JUnit \& Co. (STU) zur Vereinfachung von Unit-Tests für Java-Anwendungen. STU steht unter der Apache License 2.0 und wird von der Firma Seitenbau entwickelt. Für Tests von Datenbank-basierten Anwendungen setzt STU auf der Bibliothek DbUnit [2] auf. DbUnit ist ein Framework zum Testen von Datenbank-basierten Java-Anwendungen. 
%
%Ziel des Projekts ist es u.a., funktionale Tests durch die Spezifikation eines Java-API und einer speziellen Testdatenbeschreibungssprache so zu vereinfachen, dass eine Datenbank einfach zu Beginn des Tests – über den Test selbst – in den wohldefinierten Anfangszustand das Test Fixture versetzt werden kann. Des Weiteren soll der erwartete Datenbank-Zustand am Testende einfach auf Basis des Test Fixture beschrieben werden können. Nach Ausführung des Tests soll automatisch der über das SUT erzeugte Datenbank-Zustand mit dem erwarteten Datenbank-Zustand verglichen und somit die Korrektheit der Anwendung geprüft werden können.

%%%----

%Alle Voraussetzungen für einen Test werden dabei als sog. Test Fixture bezeichnet. 
%%
%Idealerweise soll die in einem Test Fixture beinhaltete Datenmenge für einen funktionalen Test eine hohe Testabdeckung bieten, dabei aber so klein und übersichtlich wie möglich sein. Somit sind die Testdaten einfacher zu verstehen und für eine große Anzahl von funktionalen Tests zu verwenden.
%
%
%
%Ein üblicher Ansatz, ein SUT in Verbindung mit einer Datenbank zu testen, ist das Testmuster Back Door Manipulation \cite{XUNIT_TEST_PATTERNS}.
%
%Alle Voraussetzungen für einen Test werden dabei als \emph{Test Fixture} bezeichnet.
%%
%Ein üblicher Ansatz, ein SUT in Verbindung mit einer Datenbank zu testen, ist das Testmuster Back Door Manipulation \cite{XUNIT_TEST_PATTERNS}. Im Setup wird die Datenbank über direkten Zugriff an dem SUT vorbei in den Anfangszustand des Test Fixture gebracht. Anschließend können im Exercise-Schritt die zu testenden Operationen am SUT durchgeführt werden. Die Überprüfung, ob sich das SUT richtig verhalten hat, findet im Verify-Schritt statt. Dabei kann der Zustand der Datenbank mit dem erwarteten Zustand des SUT verglichen, ebenfalls über die Test-eigene Datenbankverbindung. 
%%Abschließend können im vierten optionalen Schritt, Teardown, noch Aufräumarbeiten in der Datenbank implementiert sein.

%%%%%---


Softwaretests sind ein wichtiger Baustein für die Qualitätssicherung von Softwareprojekten. Für Tests von Datenbank-basierten Anwendungen müssen u.a.~Testdaten für die Datenbank spezifiziert werden, auf deren Basis das Verhalten der zu testenden Software 
%(System Under Test, SUT) 
geprüft werden kann.
%
%
Die Spezifikation dieser Testdaten ist leider i.A.~sehr umfangreich und komplex und somit aufwändig und fehleranfällig.
%
%Bei datenbank-basierten Anwendungen sind die Testdaten meist sehr umfangreich und komplex.
%
Die Komplexität ergibt sich v.a.~aus der Beschreibung der Beziehungen zwischen den einzelnen Datensätzen.
%im Test Fixture.
%
Diese unterliegen einer Menge komplexer fachlicher Regeln, die sich aus dem Domänen-Modell und der Geschäftslogik der Anwendung ergeben.
%
Besonders bei Systemen mit großen oder komplexen Datenbank-Schemata kann ein Testdaten-Set schnell unübersichtlich werden.


Übergreifendes Ziel der hier beschriebenen Arbeit \cite{MT:Moll:2013} war es, die Spezifikation von Testdaten für Datenbank-basierte Java-Anwendungen zu vereinfachen.
%
Zum einen wurde hierzu eine geeignete Domänen-spezifische Sprache (DSL) für Testdaten entwickelt.
%
Die DSL erlaubt eine  übersichtliche Spezifikation von Testdaten, ist einfach zu nutzen und integriert sich in gängige Entwicklungsumgebungen.
%
Zum anderen wurde ein Generator zur automatischen Erzeugen von Testdaten implementiert. 
%
Das Ziel hierbei war es, mit möglichst wenig Datensätzen viele Grenzfälle bei Beziehungen abzudecken.
%
Basis der Entwicklungsarbeiten war die Java-Bibliothek Simple Test Utils for JUnit \& Co. (STU) zur Vereinfachung von Unit-Tests für Java-Anwendungen. STU steht unter der Apache License 2.0 und wird federführend von der Firma Seitenbau entwickelt. Für Tests von Datenbank-basierten Anwendungen setzt STU auf der Bibliothek DbUnit auf. 


\begin{figure}[tb]
	\begin{center}
		\includegraphics[width=12.5cm]{images/ansatz.png}
		\caption{\label{ansatz}Überblick über den gewählten Ansatz.}
	\end{center}
   %\vspace{-0.5cm}  %%% BAD HACK JW
\end{figure}

Abbildung \ref{ansatz} gibt einen Überblick über den im Projekt gewählten Ansatz.
%
Ausgangspunkt ist eine speziellen Beschreibung des relationalen Datenbankschemas (Details siehe \cite{MT:Moll:2013}).
%
Diese kann mittels eines Tools (nicht in der Abb.~dargestellt) manuell erstellt bzw. aus einer existierenden Datenbank extrahiert und ergänzt werden.
%
Aus der Beschreibung des Datenbankschemas wird die schema-abhängige Testdaten-DSL generiert. 
%
Diese DSL kann dann von den Software-Testern genutzt werden, um verschiedene Testdaten-Sets zu beschreiben und diese mittels STU in ihre Unit-Tests einzubinden.
%
Die mittels der DSL spezifizierten Testdaten werden dabei durch das STU-Frameworks automatisch in die Datenbank eingespielt (Backdoor-Manipulation \cite{XUNIT_TEST_PATTERNS}). 
%
Ausgehend von der Schema-Beschreibung können in der DSL beschriebene Test\-da\-ten-Sets automatisch generiert werden. Die generierten Testdaten können ggfls.~vor Verwendung noch modifiziert werden.
%
%
%
%Details und ausführliche Code-Beispiele zur Verwendung sind in \cite{MT:Moll:2013} zu finden.





	






\clearpage

\section{Testdaten-DSL}


%\textbf{Aus FORUM Artikel}
%
%Die Spezifikation der Testdaten soll mit Hilfe einer Domänenspezifische Sprache (DSL) und des Java-API erfolgen. Eine DSL ist eine formale Sprache, die speziell für ein bestimmtes Problemfeld, die sogenannte Domäne, entworfen und implementiert wird. Der Entwurf einer DSL soll einen hohen Grad an Problemspezifität erreichen, d.h. die Sprache soll alle Probleme der Domäne darstellen können und nichts darüber hinaus. Dadurch ist die Sprache durch Domänenspezialisten, in unserem Fall z.B. die Software-Tester, ohne besonderes Zusatzwissen leicht nutzbar.
%
%Im Projektkontext bedeutet domänenspezifisch, dass die DSL die grundlegenden Konzepte der Datenbank-Strukturierung (in unserem Fall das relationale Datenmodell) unterstützen muss, aber auch vom fachlichen Datenbankschema abhängig ist, das dem SUT zugrundeliegt.  D.h. für jede zu testende Anwendung wird aus einer erweiterten Spezifikation des zugrundeliegenden Datenbankschemas eine spezielle DSL erzeugt, in der die Testdaten dann durch die Software-Tester bzw. die Domänenexperten beschrieben werden können (siehe Abb. 3).

Es wurden verschiedene Ansätze zur Entwicklung einer DSL für Testdaten untersucht. Der Fokus lag u.a. auf der Fachlichkeit der Datenstruktur, der typsicheren Beschreibung der Testdaten und der einfachen Spezifikation von Beziehungen zwischen Entitäten. Untersucht wurden verschiedene XML-basierte Darstellungen, wie z.B. in DbUnit benutzt, programmatische Spezifikationen und verschiedene tabellarischen Beschreibungsformen für die Testdaten. 
%
Nach einer Evaluation wurde eine tabellarische Beschreibungsform gewählt, die über das STU-Framework genutzt werden kann. Diese Art der Testdatenmodellierung ist übersichtlich, syntaktisch einfach und kann von einer IDE wie Eclipse unterstützt werden. Die grundlegende Idee für die tabellarische Darstellung stammt vom Testframework Spock \cite{Spock}.
%
Die EBNF der DSL ist in \cite{MT:Moll:2013} zu finden.



\begin{figure}[tb]
\begin{lstlisting}[caption=Mittels DSL beschriebenes Testdaten-Set (Table Builder API)., style=java, label=listing:dsl]
class BookDatabaseGroovyDataSet extends BookDatabaseBuilder
{
  <def> tables() {
    `buchTable`.rows {
      @REF@            | @name@
      @CLEANCODE@      | "Clean Code"      
      @EFFECTIVEJAVA@  | "Effective Java"  
      @DESIGNPATTERNS@ | "Design Patterns" 
    }
    `verlagTable`.rows {
      @REF@           | @name@
      @PRENTICE@      | "Prentice Hall International"
      @ADDISONWESLEY@ | "Addison-Wesley"
    }
    `autorTable`.rows {
      @REF@       | @vorname@     | @nachname@
      @UNCLEBOB@  | "Robert C." | "Martin"
      @BLOCH@     | "Joshua"    | "Bloch"
      @GAMMA@     | "Erich"     | "Gamma"
      @HELM@      | "Richard"   | "Helm"
      @JOHNSON@   | "Ralph"     | "Johnson"
      @VLISSIDES@ | "John"      | "Vlissides"    
    }
  }

  <def> relations() {
    @PRENTICE@.verlegt(@CLEANCODE@)
    @ADDISONWESLEY@.verlegt(@EFFECTIVEJAVA@, @DESIGNPATTERNS@)
    @CLEANCODE@.geschriebenVon(@UNCLEBOB@)
    @EFFECTIVEJAVA@.geschriebenVon(@BLOCH@)
    @DESIGNPATTERNS@.geschriebenVon(@GAMMA@, @HELM@, @JOHNSON@, @VLISSIDES@)
  }
}
\end{lstlisting}
\end{figure}


Listing \ref{listing:dsl} zeigt beispielhaft einen Auszug aus einer DSL-Beschreibung für Testdaten einer Anwendung zur Verwaltung von Büchern (Datenbank-Schema siehe Abb.~\ref{database}). In der tabellarischen Darstellung (\texttt{tables}) enthält die erste Zeile die Spaltennamen der Tabelle, die anderen Zeilen enthalten die einzufügenden Daten. Die erste Spalte einer Datenzeile enthält jeweils einen symbolischen Namen (\texttt{REF}) für den Tabelleneintrag, der zur Referenzierung und somit Spezifikation von Beziehungen (\texttt{relations}) zwischen Datensätzen genutzt werden kann.


Die Implementierung der Testdaten-DSL basiert dabei auf Groovy, einer dynamisch typisierten Programmiersprache für die Java Virtual Machine. Die DSL kann eingebettet zusammen mit Java in den Tests, z.B. mit JUnit, genutzt werden. 
...
%

Details zur Implementierung sind in \cite{MT:Moll:2013} zu finden.

\textbf{ToDo: Evtl. nicht mehr notwendig:} Definition eines Datenbank-Schemas, Details Generierung der DSL sowie Unit-test Beispiele etc. ...



%\textbf{Outline}
%
%
%	\subsection{Sprach-Definition}
%
%	\subsection{Anwendung der Sprache}
%	
%	- Definieren eines Datenbank-Modells
%	
%	- Generieren der DSL
%	
%	- Unit-Test-Beispiel
%	
%
%	\subsection{Implementierung und Evaluation}
%	
%	- Implementiert auf Java-Basis mit Groovy
%	
%	- Apache 2 Lizenz
%	
%	- http://github.com/seitenbau/stu
%	
%	- Implementierungsdetails in Thesis
%	
%	- Einsatz in Beispielen und realen Projekten geprüft
	


\clearpage

\section{Der Test-Daten-Generator}
Idee hinter dem Algorithmus

	\subsection{Beschreibung des Algorithmus}

	\begin{figure}[htb]
		\begin{center}
			\includegraphics[width=8cm]{images/database.png}
			\caption{\label{database}Datenbank-Diagramm}
		\end{center}
	\end{figure}
	

	

	\subsection{Implementierung und Evaluation}



\clearpage

\section{Fazit}

Die in dieser Arbeit vorgestellten Konzepte und die daraus resultierende Software wurden in das vorhandene STU-Framework integriert.
%
Der  Code steht  unter der Apache License 2.0 unter \texttt{https://github.com/seitenbau/stu/}  zur Verfügung. 
%

Die entwickelte Testdaten-DSL wurde bereits in der Qualitätssicherung von mehreren produktiven Softwareprojekten  eingesetzt.
%
Der Spezifikationsaufwand und die Fehlerrate konnte  im Vergleich zur  früheren Vorgehensweise  deutlich reduziert werden.
%
Der Testdatengenerator wurde dabei  auf  Datenbankschemata mit teilweise mehr als 80 Tabellen angewandt.
%
Der  Testdatengenerator konnte in jedem Fall  einen konsistenten, übersichtlichen Testdaten-Set erzeugen, der eine sehr gute Startbasis für die Anwendungstests ergab.
%Die erzeugten Testdaten-Sets boten eine sehr gute Startbasis für die Anwendungstests.

%Der  Code steht  unter \texttt{https://github.com/seitenbau/stu/} unter der Apache License 2.0 zur Verfügung. 
%
%Details zur Implementierung und ausführliche Code-Beispiele zur Verwendung sind in \cite{MT:Moll:2013} zu finden.



\bibliography{GI-IT-TD}

\end{document}



