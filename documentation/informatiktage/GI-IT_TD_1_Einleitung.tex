\section{Einleitung und Motivation}

Softwaretests sind ein wichtiger Baustein f�r die Qualit�tssicherung von modernen Softwareprojekten.
F�r Tests werden Testdaten spezifiziert. Auf deren Basis wird das Verhalten der Software gepr�ft.
Bei Datenbank-basierten Anwendungen werden Testdaten in der Regel sehr umfangreich und komplex.

Die Komplexit�t ergibt sich aus der Beschreibung von Beziehungen zwischen den einzelnen
Datens�tzen. Besonders bei Systemen mit komplexen Datenbank-Schemata kann ein
Testdaten-Modell schnell un�bersichtlich werden.

F�r den Tester sind un�bersichtliche Testdaten aus verschiedenen Gr�nden ein Problem.
Einerseits machen sie die Pflege fehleranf�llig. Andererseits ist es schwer,
die modellierten Daten zu erfassen und zu verstehen. Tester w�nschen deshalb h�ufig Daten,
die f�r mehrere Tests nutzbar sind. Dadurch reicht es aus, nur ein oder zumindest
wenige Daten-Modelle zu verstehen und zu pflegen.

Die im Rahmen einer Master Thesis [Quelle] entwickelte Modellierungssprache f�r Testdaten erlaubt
eine �bersichtliche Modellierung von Testdaten. Sie ist einfach zu nutzen und integriert sich
in g�ngige Entwicklungsumgebungen.

Dar�ber hinaus wurde ein Algorithmus f�r die Generierung von Testdaten konzipiert. Das Ziel des
Generators ist es, mit wenigen Datens�tzen viele Grenzf�lle bei Beziehungen abzudecken.


