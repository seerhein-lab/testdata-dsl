\section{Testdaten-DSL}


Es wurden verschiedene Ansätze zur Entwicklung einer DSL für Testdaten untersucht. Der Fokus lag u.a.~auf der Fachlichkeit der Datenstruktur, der typsicheren Beschreibung der Testdaten und der einfachen Spezifikation von Beziehungen zwischen Entitäten. Untersucht wurden verschiedene XML-basierte Darstellungen, wie z.B. in DbUnit benutzt, programmatische Spezifikationen und  tabellarische Beschreibungsformen. 
%
Nach einer Evaluation wurde eine tabellarische Beschreibungsform gewählt.
%
%, die über das STU-Framework genutzt werden kann. 
%
Diese Art der Testdatenmodellierung ist übersichtlich und  syntaktisch einfach. Die grundlegende Idee 
%
%für die tabellarische Darstellung 
%
stammt vom Testframework Spock \cite{Spock}.
%
Die EBNF der DSL ist in \cite{MT:Moll:2013} zu finden.



\begin{figure}[tb]
\begin{lstlisting}[caption=Beispiel eines mittels DSL beschriebenen Testdaten-Sets., style=java, label=listing:dsl]
class BookDatabaseGroovyDataSet extends BookDatabaseBuilder {
  <def> tables() {
    `buchTable`.rows {
      @REF@            | @name@
      @CLEANCODE@      | "Clean Code"      
      @EFFECTIVEJAVA@  | "Effective Java"  
      @DESIGNPATTERNS@ | "Design Patterns" }
    `verlagTable`.rows {
      @REF@           | @name@
      @PRENTICE@      | "Prentice Hall International"
      @ADDISONWESLEY@ | "Addison-Wesley" }
    `autorTable`.rows {
      @REF@       | @vorname@     | @nachname@
      @UNCLEBOB@  | "Robert C." | "Martin"
      @BLOCH@     | "Joshua"    | "Bloch"
      @GAMMA@     | "Erich"     | "Gamma"
      @HELM@      | "Richard"   | "Helm"
      @JOHNSON@   | "Ralph"     | "Johnson"
      @VLISSIDES@ | "John"      | "Vlissides"  }
  }
  <def> relations() {
    @PRENTICE@.verlegt(@CLEANCODE@)
    @ADDISONWESLEY@.verlegt(@EFFECTIVEJAVA@, @DESIGNPATTERNS@)
    @CLEANCODE@.geschriebenVon(@UNCLEBOB@)
    @EFFECTIVEJAVA@.geschriebenVon(@BLOCH@)
    @DESIGNPATTERNS@.geschriebenVon(@GAMMA@, @HELM@, @JOHNSON@, @VLISSIDES@) 
  }
}
\end{lstlisting}
\end{figure}


Listing \ref{listing:dsl} zeigt beispielhaft die Testdaten-DSL für eine Bücherverwaltung (Datenbankschema siehe Abb.~\ref{generiert} oben). In der tabellarischen Darstellung (\texttt{tables}) enthält die erste Zeile die Spaltennamen der Tabelle, die anderen Zeilen enthalten die einzufügenden Daten. Die erste Spalte einer Datenzeile enthält jeweils einen symbolischen Namen (\texttt{REF}) für den Tabelleneintrag, der zur Referenzierung und somit Spezifikation von Beziehungen (\texttt{relations}) zwischen Datensätzen genutzt werden kann.

Die Implementierung der Testdaten-DSL basiert auf 
%
der dynamischen Programmiersprache Groovy 
%
%\footnote{http://groovy.codehaus.org/} 
und verwendet Laufzeit-Metaprogrammierung in Verbindung mit Operator-Überladen \cite{MT:Moll:2013}. Die DSL kann eingebettet zusammen mit Java in den Tests (z.B.~mit JUnit) genutzt werden und integriert sich sehr gut in gängige IDEs wie Eclipse. Die Spaltennamen sind in der DSL definiert, so dass Autovervollständigung unterstützt wird. 
%
%Die zur DSL generierte JavaDoc enthält Beispiele und Informationen zu den Tabellen. 
%
Über die REF-Namen können Beziehungen typsicher modelliert und konkrete Werte abgefragt werden. 
%
Details zur Implementierung, zur Generierung der DSL für ein Datenbankschema und zur Nutzung der DSL in Softwaretests mit STU sind in \cite{MT:Moll:2013} zu finden.


