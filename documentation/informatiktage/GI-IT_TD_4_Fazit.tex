\section{Fazit}

Die in dieser Arbeit vorgestellten Konzepte und die daraus resultierende Software wurden in das vorhandene STU-Framework integriert.
%
Der  Code steht  unter der Apache License 2.0 unter \texttt{https://github.com/seitenbau/stu/}  zur Verfügung. 
%

Die entwickelte Testdaten-DSL wurde bereits in der Qualitätssicherung von mehreren produktiven Softwareprojekten  eingesetzt.
%
Der Spezifikationsaufwand und die Fehlerrate konnte  im Vergleich zur  früheren Vorgehensweise  deutlich reduziert werden.
%
Der Testdatengenerator wurde dabei  auf  Datenbankschemata mit teilweise mehr als 80 Tabellen angewandt.
%
Der  Testdatengenerator konnte in jedem Fall  einen konsistenten, übersichtlichen Testdaten-Set erzeugen, der eine sehr gute Startbasis für die Anwendungstests ergab.
%Die erzeugten Testdaten-Sets boten eine sehr gute Startbasis für die Anwendungstests.

%Der  Code steht  unter \texttt{https://github.com/seitenbau/stu/} unter der Apache License 2.0 zur Verfügung. 
%
%Details zur Implementierung und ausführliche Code-Beispiele zur Verwendung sind in \cite{MT:Moll:2013} zu finden.